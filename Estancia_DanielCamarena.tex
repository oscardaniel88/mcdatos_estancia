\documentclass[a4paper,11pt]{article}
\usepackage[utf8]{inputenc}
\usepackage{lmodern}
\usepackage{amsmath}
\usepackage{enumitem}
\title{\Huge Modelo de Predicción de Demanda y Pricing Dinámico}
\author{Oscar Daniel Camarena Gómez}
\date{}
\begin{document}
\maketitle
\pagenumbering{gobble}
\newpage
\pagenumbering{arabic}
\section{Introducción}
\paragraph {CityExpress} ~ \\
CityExpress es una cadena de hoteles 100\% mexicana enfocada al turismo de negocios. Actualmente cuenta con 140 hoteles en México y Latinoamérica (Colombia, Costa Rica y Chile). CityExpress cuenta con hoteles pensados para el viajero de negocios, sus instalaciones son prácticas, la tarifa es baja y ofrece servicios limitados pero valiosos durante un viaje de negocios, por ejemplo:
\begin{itemize}[noitemsep]
\item Internet gratuito en las habitaciones
\item Desayuno de cortesía
\item Transporte gratuito 10 km a la redonda
\item Servicio estandarizado con bajo costo
\end{itemize}
CityExpress cuenta con 4 marcas, cada una de ellas enfocada a distintos segmentos de mercado:
\begin{description}
\item [$\bullet$ CityExpress:] Marca emblema, enfocada al viajero de negocios. Los hoteles están ubicados cerca de los centros de negocios de las distintas ciudades en la República Mexicana.
\item [$\bullet$ CityExpress Plus:] Hoteles ubicados en las principales ciudades de la República Mexicana (CDMX, Guadalajara, Monterrey) y en Latinoamérica (Colombia). Ofrecen diseños vanguardistas y mayor espacio en las habitaciones.
\item [$\bullet$ CityExpress Suites:] Marca enfocada en los viajeros de larga estancia. Sus instalaciones cuentan con cocineta, estancia y habitaciones de mayor tamaño.
\item [$\bullet$ CityExpress Junior:] Hoteles de menor precio enfocados al turismo con presupuesto limitado ofreciendo un producto de calidad y con servicio estandarizado.
\end{description}
\paragraph {Ubicación} ~ \\
Típicamente un hotel de CityExpress está ubicado cerca de zonas industriales o zonas con fuerte actividad económica, es decir, en el caso de las grandes ciudades encontraremos hoteles CityExpress cerca de una zona donde hay alta densidad de locales comerciales, oficinas corporativas o centros de negocios. En el caso de ciudades mas pequeñas o con actividades económicas específicas encontraremos hoteles cerca de los centros económicos de cada una de las ciudades, este hecho permite que la cadena agrupe a sus hoteles en distintos \textbf{corredores} dependiendo de la actividad de las zonas económicas donde se encuentran. A continuación se presentan los distintos corredores:
\begin{itemize}[noitemsep]
\item Energético
\item Exportacion-Agricultura
\item Franquicia
\item Internacional
\item Manufactura
\item Maquila
\item Mineria
\item Servicios
\end{itemize}
\paragraph {Hoteles} ~ \\
Las propiedades de Hoteles CityExpress siguen un estándar desde su construcción hasta en su operación diaria. En cuanto al estándar de construcción las propiedades generalmente cuentan con 6 plantas, 120 habitaciones, un lobby para la recepción de huéspedes y áreas comunes; por ejemplo, área designada para fumar, desayunador, salas de juntas, salones de eventos y alberca (en algunas propiedades). 
En cuanto a las habitaciones el tamaño puede variar por marca, sin embargo ofrecen los mismos servicios independientemente de ella: 
\begin{itemize}[noitemsep]
\item Internet Gratuito
\item Área de Trabajo
\item Lavandería
\item Aire Acondicionado
\item Pantalla de televisión
\item Desayuno de cortesía
\end{itemize}
\paragraph {Crecimiento} ~ \\
Hoteles CityExpress fue fundada en 2002 con su primer hotel construido en Saltillo. A partir de ese entonces se ejecutó un ambicioso plan de crecimiento en el cuál se impuso la meta de abrir un nuevo hotel cada 6.8 semanas. Esto llevo a que la cadena de hoteles contara con más de 140 propiedades en un lapso de 16 años.
Hoy en día Hoteles CityExpress cuenta con propiedades en 34 de los 36 estados de la República Mexicana y 5 mas en Latinoamérica: 1 en Costa Rica, 1 en Chile, 3 en Colombia. El plan de crecimiento sigue siendo ambicioso y la cadena espera contar con 180 propiedades para el 2020.
\section{Descripción del Problema}
\subsection{Gestión de las propiedades}
Hoteles CityExpress cuenta con un modelo de gestión comercial de sus propiedades ejecutado desde sus oficinas centrales ubicadas en la Ciudad de México. Este modelo de gestión se enfoca en optimizar la venta de los cuartos de las distintas propiedades, buscando siempre maximizar el ingreso obtenido por estas ventas.
Para poder entender mejor el modelo de gestión comercial debemos estudiar antes los canales distribución que utiliza para la venta de sus productos así como sus distintos segmentos de mercado. Posteriormente presentaremos las principales variables utilizadas para calificar el desempeño de las propiedades así como las acciones que puede tomar cada uno de los gerentes de las propiedades para poder cambiar el desempeño del hotel.
\paragraph{Canales de Venta} ~\\
CityExpress cuenta con un modelo de venta estandarizado para todas sus propiedades. Este modelo de venta contempla distintos canales de distribución que ayudan al grupo a poder incrementar la oferta de su producto llegando a distintos segmentos de mercado.
Es importante mencionar que cada uno de los canales de venta tienen un costo asociado, es decir, del ingreso que se perciba por cada cuarto vendido se debe tomar en cuenta el costo de cada canal para poder calcular la utilidad bruta de la venta, por eso es. importante conocer cómo opera cada uno de ellos.
Dentro de CityExpress existen dos tipos de canales de venta: \textbf{Canales de venta tradicionales} y \textbf{Canales de venta electrónicos}. Los canales de venta tradicionales, son aquellos en los cuales se requiere la intervención de un agente de ventas, los electrónicos permiten que el cliente final realice su reservación sin requerir la intervención del agente:
\paragraph{\textbf{Canales de venta tradicionales}}
\begin{itemize}[noitemsep]
\item Centro de Contacto (Call Center)
\item Hotel
\end{itemize}
\paragraph{\textbf{Canales de venta electrónicos}}
\begin{itemize}[noitemsep]
\item App (IOS y Android)
\item Portal Cliente Frecuente (City Premios)
\item Motor de reservaciones corporativas (CityAccess)
\item Motor de reservaciones (www.cityexpress.com.mx)
\item Agencias de viajes en línea (OTAs)
\end{itemize}
\paragraph{Segmentos de Mercado} ~\\
Los segmentos de mercado son un punto medular para poder entender el modelo de gestión en CityExpress ya que estos permiten al área comercial poder establecer una estrategia de venta en cada uno de los canales fijando precios en las distintas tarifas para poder maximizar la ocupación del hotel y el ingreso por cuartos vendidos. 
Típicamente un hotel cuenta con un conjunto de tarifas las cuales son ofrecidas a diferentes segmentos de mercado en diferentes canales. Por ejemplo, un huésped que viaja por placer y reserva por la página web de CityExpress muy probablemente recibirá una tarifa distinta al huésped que viaja por negocios y reservó por el centro de contacto.
A continuación se presentan los distintos segmentos de mercado manejados por CityExpress:
\paragraph{\textbf{Negocios}}
\begin{itemize}[noitemsep]
\item Directos
\item Convenios
\item Viajero Frecuente
\item Corporativo
\item Consorcios
\item Promociones
\item Otros
\end{itemize}
\paragraph{\textbf{Placer}}
\begin{itemize}[noitemsep]
\item Directos
\item Fin de Semana
\item Otros
\item Promociones
\end{itemize}
\paragraph{\textbf{Mayoreo}}
\begin{itemize}[noitemsep]
\item Agencias Minoristas
\item OTAs
\item Lineas Aéreas / Tripulaciones
\item Agencias Mayoristas
\item Otros
\end{itemize}
\paragraph{\textbf{Grupos}}
\begin{itemize}[noitemsep]
\item Grupos, Ferias, Congresos y Convenciones
\item Sociales
\item Juntas
\item Deportes
\end{itemize}
\paragraph{\textbf{Otros}}
\begin{itemize}[noitemsep]
\item Uso Casa
\item Cortesías
\item Intercambio
\item Empleados
\item Industria Turística
\item Otros
\end{itemize}
\section{Metodología Propuesta}
\section{Resultados}
\section{Referencias}
\end{document}