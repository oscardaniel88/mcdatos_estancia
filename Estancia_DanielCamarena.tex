\documentclass[a4paper,11pt]{article}
\usepackage[utf8]{inputenc}
\usepackage{lmodern}
\usepackage{amsmath}
\usepackage{enumitem}
\title{\Huge Modelo de Predicción de Demanda y Pricing Dinámico}
\author{Oscar Daniel Camarena Gómez}
\date{}
\begin{document}
\maketitle
\pagenumbering{gobble}
\newpage
\pagenumbering{arabic}
\section{Introducción}
\paragraph {CityExpress} ~ \\
CityExpress es una cadena de hoteles 100\% mexicana enfocada al turismo de negocios. Actualmente cuenta con 140 hoteles en México y Latinoamérica (Colombia, Costa Rica y Chile). CityExpress cuenta con hoteles pensados para el viajero de negocios, sus instalaciones son prácticas, la tarifa es baja y ofrece servicios limitados pero valiosos durante un viaje de negocios, por ejemplo:
\begin{itemize}[noitemsep]
\item Internet gratuito en las habitaciones
\item Desayuno de cortesía
\item Transporte gratuito 10 km a la redonda
\item Servicio estandarizado con bajo costo
\end{itemize}
CityExpress cuenta con 4 marcas, cada una de ellas enfocada a distintos segmentos de mercado:
\begin{description}
\item [$\bullet$ CityExpress:] Marca emblema, enfocada al viajero de negocios. Los hoteles están ubicados cerca de los centros de negocios de las distintas ciudades en la República Mexicana.
\item [$\bullet$ CityExpress Plus:] Hoteles ubicados en las principales ciudades de la República Mexicana (CDMX, Guadalajara, Monterrey) y en Latinoamérica (Colombia). Ofrecen diseños vanguardistas y mayor espacio en las habitaciones.
\item [$\bullet$ CityExpress Suites:] Marca enfocada en los viajeros de larga estancia. Sus instalaciones cuentan con cocineta, estancia y habitaciones de mayor tamaño.
\item [$\bullet$ CityExpress Junior:] Hoteles de menor precio enfocados al turismo con presupuesto limitado ofreciendo un producto de calidad y con servicio estandarizado.
\end{description}
\paragraph {Ubicación} ~ \\
Típicamente un hotel de CityExpress está ubicado cerca de zonas industriales o zonas con fuerte actividad económica, es decir, en el caso de las grandes ciudades encontraremos hoteles CityExpress cerca de una zona donde hay alta densidad de locales comerciales, oficinas corporativas o centros de negocios. En el caso de ciudades mas pequeñas o con actividades económicas específicas encontraremos hoteles cerca de los centros económicos de cada una de las ciudades, este hecho permite que la cadena agrupe a sus hoteles en distintos \textbf{corredores} dependiendo de la actividad de las zonas económicas donde se encuentran. A continuación se presentan los distintos corredores:
\begin{itemize}[noitemsep]
\item Energético
\item Exportacion-Agricultura
\item Franquicia
\item Internacional
\item Manufactura
\item Maquila
\item Mineria
\item Servicios
\end{itemize}
\paragraph {Hoteles} ~ \\
Las propiedades de Hoteles CityExpress siguen un estándar desde su construcción hasta en su operación diaria. En cuanto al estándar de construcción las propiedades generalmente cuentan con 6 plantas, 120 habitaciones, un lobby para la recepción de huéspedes y áreas comunes; por ejemplo, área designada para fumar, desayunador, salas de juntas, salones de eventos y alberca (en algunas propiedades). 
En cuanto a las habitaciones 

\section{Descripción del Problema}
\section{Metodología Propuesta}
\section{Resultados}
\section{Referencias}
\end{document}