\documentclass[a4paper,11pt]{article}
\usepackage[utf8]{inputenc}
\usepackage{lmodern}
\usepackage{amsmath}
\usepackage{enumitem}
\title{\Huge Modelo de Predicción de Demanda y Pricing Dinámico}
\author{Oscar Daniel Camarena Gómez}
\date{}
\begin{document}
\maketitle
\pagenumbering{gobble}
\newpage
\pagenumbering{arabic}
\section{Introducción}
\paragraph {} ~ \\
CityExpress es una cadena de hoteles 100\% mexicana enfocada al turismo de negocios. Actualmente cuenta con 140 hoteles en México y Latinoamérica (Colombia, Costa Rica y Chile). CityExpress cuenta con hoteles pensados para el viajero de negocios, sus instalaciones son prácticas, la tarifa es baja y ofrece servicios limitados pero valiosos durante un viaje de negocios, por ejemplo:
\begin{description}
\item [$\bullet$ Internet gratuito en las habitaciones]
\item [$\bullet$ Desayuno de cortesía]
\item [$\bullet$ Transporte gratuito 10 km a la redonda]
\item [$\bullet$ Servicio estandarizado con bajo costo]
\end{description}
CityExpress cuenta con 4 marcas, cada una de ellas enfocada a distintos segmentos de mercado:
\begin{description}
\item [$\bullet$ CityExpress:] Marca emblema, enfocada al viajero de negocios. Los hoteles están ubicados cerca de los centros de negocios de las distintas ciudades en la República Mexicana.
\item [$\bullet$ CityExpress Plus:] Hoteles ubicados en las principales ciudades de la República Mexicana (CDMX, Guadalajara, Monterrey) y en Latinoamérica (Colombia). Ofrecen diseños vanguardistas y mayor espacio en las habitaciones.
\item [$\bullet$ CityExpress Suites:] Marca enfocada en los viajeros de larga estancia. Sus instalaciones cuentan con cocineta, estancia y habitaciones de mayor tamaño.
\item [$\bullet$ CityExpress Junior:] Hoteles de menor precio enfocados al turismo con presupuesto limitado ofreciendo un producto de calidad y con servicio estandarizado.
\end{description}
\section{Descripción del Problema}
\section{Metodología Propuesta}
\section{Resultados}
\section{Referencias}
\end{document}