\chapter{Marco teórico}
\label{ch:marco teorico}

En este capítulo se introducirá y discutirá el trabajo previo realizado y los conceptos necesarios para comprender las distintas técnicas y enfoques utilizados para la construcción de un módelo de predicción de demanda y maximización de ingresos así como el impacto que los resultados tienen dentro de la industria hotelera. Se dará una breve explicación del proceso de gestión de una propiedad mencionando cuales son los principales indicadores de control del proceso, así como las variables estudiadas para la toma de decisiones.

\section*{Gestion de una propiedad}

El proceso de gestión de propiedades o de hoteles comprende una serie de actividades que tienen como objetivo el garantizar la rentabilidad de un proyecto de arrendamiento. Dentro de las actividades que se llevan a cabo están las siguientes:
\begin{itemize}
  \item Gestión de \emph{canales de venta}
  \item Gestión de \emph{segmentos de mercado}
  \item Gestión de \emph{precios por tarifa}
  \item Análisis del \emph{comportamiento de la plaza}
  \item Análisis de \emph{indicadores principales}
\end{itemize}

A continuación profundizaremos en cada una de las actividades que forman parte de este proceso.

\subsection*{Gestión de canales de venta}

El equipo administrativo del hotel o propiedad, deben analizar diariamente el comportamiento de sus canales de venta o distribución. Típicamente un hotel tiene un catálogo estándar de canales de distribución:
\begin{itemize}
  \item Sitio Web 
  \item Call Center / FrontDesk
  \item Agencias de viajes en línea
  \item Globalizadores ~ Agencias de viajes
\end{itemize}

Cada uno de los canales de venta van dirigidos a un segmento de mercado en específico y lo que el equipo de administración debe hacer es asegurar que la demanda del hotel se mantenga lo suficientemente alta para poder asegurar la continuidad operativa del negocio. Para lograr esto se genera una estrategia de distribución, en la cuál se designa parte del inventario a cada uno de los canales y se fija un precio a cada una de las tarifas a ofertar. Los precios para un producto en específico tienden a variar entre los diferentes canales y esto tiene que ver con el costo asociado a la operación de cada uno de los canales. Por ejemplo, el sitio web típicamente es un canal de reservación propio del hotel, lo que significa que las ventas generadas por este canal son libres de comisiones; por el contrario, las agencias de viajes en línea tienden a cobrar una comisión por la venta generada por esos canales.

Lo que la administración del hotel debe conseguir es poder tener la mezcla de ventas entre sus canales que maximice el ingreso, de tal forma que los costos generados por las ventas no \emph{canibalicen} las ganancias de la propiedad. Es con esta restricción donde se descartaría una asignación total del inventario a las agencias de viajes en línea, ya que la comisión de este canal puede ser de hasta el 25\% del total de la venta, lo cual podría representar una pérdida de hasta $\frac{1}{4}$ de las ganancias que se pudieron haber obtenido. Si por el contrario asignaramos todo el inventario disponible a los canales propios del hotel, si bien no se paga una comisión, la demanda generada por estos canales es inferior a la demanda generada por las agencias de viajes en línea.

\subsection*{Gestión de segmentos de mercado}

La gestión de los segmentos de mercado es similar a la gestión de los canales de venta ya que estos estan relacionados a los canales de distribución. 

Las propiedades típicamente distinguen segmentan a sus clientes en las siguientes categorías:

\begin{itemize}
  \item \emph{Clientes directos:} Aquellos clientes que reservan en algún canal directo del hotel
  \item \emph{Clientes de negocios:} Aquellos clientes que pertenecen a una empresa que tiene alguna tarifa convenida con el hotel. Típicamente esta tarifa esta sujeta a una producción de cuartos durante un periodo
  \item \emph{Clientes de mayoreo:} Aquellos clientes que reservan utilizando un canal mayorista; puede ser una agencia de viajes en línea o mediante un agente de viajes.
  \item \emph{Grupos:} Aquellos clientes que llegan a la propiedad como parte de un grupo.
\end{itemize}

Para gestionar eficientemente los segmentos de mercado se debe crear una estrategia de asignación de precios a las diferentes tarifas que van dirigidas a cada uno de ellos. Por ejemplo, un \emph{cliente directo} que llega al hotel sin reservación, es propenso a pagar un precio mas alto por una habitación disponible, al contrario de un cliente que viene como parte de un grupo que reservó con mayor tiempo de anticipación con una tarifa mucho mas baja. El equipo administrativo debe cuidar los segmentos de mercado que ocupan la propiedad ya que el tener un grupo muy grande ocupándola significa muchas veces tener un hotel lleno a una tarifa muy baja y con mayor desgaste en las habitaciones, y por el otro lado, no siempre es posible llenar un hotel con clientes que llegan sin reservación los cuales están dispuestos a comprar una habitación a un precio mas elevado.

\subsection*{Gestión de precios por tarifa}

Una de las actividades mas importantes del proceso de gestión de una propiedad es la gestión de los precios por tarifa ya que generalmente las tarifas disponibles en un hotel están asociadas a un segmento de mercado y a un canal de reservación. Para poder realizar esta actividad se debe establecer una estrategia en la cual se define si se quiere incrementar la ocupación de la propiedad o si se quiere incrementar el \emph{RevPAR}(revenue per available room). Una vez establecida la estrategia inicial se deben modificar los precios para las tarifas que impactan los canales y segmentos donde se quiere fomentar la demanda o incrementar el precio. Si el objetivo es el primero, lo mas probable es que el cambio emplique una baja en las tarifas o creación de promociones sujetas a diversas restricciones, en cambio, si el objetivo es el segundo muy probablemente el cambio implique una alza en algunas tarifas y el cierre de algunas tarifas en donde este incremento no se pueda llevar a cabo por temas contractuales, por ejemplo las tarifas convenidas con otras empresas o clientes.

Es importante mencionar que la ejecución de esta tarea dentro del proceso de gestión depende del análisis continuo de la información histórica generada por el hotel (comportamiento de la propiedad en años anteriores en la misma fecha sujeta al estudio), conocimiento de la plaza en donde se ubica la propiedad (depende de la experiencia del equipo administrativo del hotel), información ajena al hotel (situación marco económica, eventos cercanos a la plaza, desempeño de la competencia cercana a la propiedad, etc). Muchas veces el conocer esta información implica un gran esfuerzo por parte del equipo ya que la información tiende a estar distribuída en diversos sistemas o en ocasiones deberán recabarla físicamente (visitando a su competencia o haciendo llamadas telefónicas) dejando un riesgo latente de recibir información que no es precisa.

Otra de las fuertes dependencias para la ejecución efectiva de esta tarea es contar con sistemas que distribuyan los nuevos precios en todos los canales disponibles en poco tiempo y con poco esfuerzo ya que de lo contrario no tendria sentido realizar esta labor si al final los precios no pueden ser cambiados dentro de un tiempo en el que el mercado pueda responder de acuerdo a lo planeado durante el análisis de la información. Una vez que los precios son alterados, el equipo deberá analizar nuevamente la información para asegurar que los cambios tuvieron los efectos deseados, de lo contrario, se deberá cambiar la estrategia lo antes posible para evitar afectar el desempeño de la propiedad.


\subsection*{Trabajo previo}
\subsection*{Modelos de series de tiempo}
\subsubsection*{ARIMA}
\subsubsection*{Modelos de pickup}
\subsection*{Suavizamientos de curvas}
\subsection*{Modelos de regresión}
\subsection*{Modelos de machine learning}
\subsection*{Resumen}