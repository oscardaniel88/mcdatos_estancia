\chapter{Introducción}
\label{ch:intro}
Una práctica común dentro de la industria de la hotelería es la maximización del ingreso, mejor conocida como \emph{revenue management}. Esta surge de la necesidad de poder obtener vender cada uno de los cuartos dentro del inventario al mayor precio posible y de esta manera poder maximizar el ingreso percibido por la porpiedad derivado de la venta de cuartos (El-Gayar N, et al. 2008).
La técnica de \emph{revenue management} tiene como base el análisis de información generada por la misma propiedad así como información ajena a ella para poder vender el producto adecuado, al cliente adecuado, en el momento y precio adecuados. Estas estrategias ayudan al hotelero en su gestión diaria de diversas maneras. Le permite establecer una estrategia coherente de precios a todos los niveles, controlar la distribución multicanal, realizar previsiones de la demanda, conocer mejor las necesidades y patrones de reserva de los diversos segmentos de clientes y disponer de más información sobre el mercado y el entorno que lo envuelve.
Los sistemas de \emph{revenue management} aplican principios básicos de economía para configurar los precios y el control del inventario que se pretende vender, de hecho, hay tres categorías básicas de administración de demanda que son utilizadas dentro del \emph{revenue management}:
\begin{itemize}
  \item \emph{Decisiones Estructurales:} Se decide qué formato de venta se va a utilizar; a qué segmento de mercado se va a dirigir la venta; cuáles serán las condiciones de la venta que se ofrecerán.
  \item \emph{Decisiones de Precio:} Cómo se asignará el precio para los elementos dentro del inventario; diferentes categorías de precios en el mismo inventario;  cambios en los precios del inventario en el tiempo.
  \item \emph{Decisiones de Cantidad de Inventario:} Se decide si se acepta o se rechaza una oferta de compra; cuanto inventario se destina a cada uno de los segmentos de mercado; cuando retirar un producto del mercado para venderlo posteriormente.
\end{itemize}
Tradicionalmente los hoteles utilizan los controles de decisión de cantidad de inventario como su estrategía por \emph{"default"} (Aziz, et al. 2011), sin embargo la tecnología ha estado cambiando el mercado y la manera en que un hotel vende su inventario, ya que hoy en día los costos asociados al cambio de precio del inventario y el tiempo para distribuirlos ha bajado considerablemnte gracias a la adopción de un ecosistema totalmente electrónico (Bitran, Caldentey 2003).
Antes del auge del \emph{revenue management} la aerolínea \emph{British Airways} experimentaba aplicando promociones en distintos productos dentro de su inventario para generar demanda en los asientos que sin esta promoción hubieran volado vacíos. Posteriormente la dirección de la aerolínea \emph{American Airlines} dió un paso más adelante y crearon una practica llamada \emph{yield management}, la cual se enfoca principalmente en maximizar los ingresos de cada ruta vendida mediante un control delinventario basado en análisis de la información histórica disponible. \emph{American Airlines} hizo fuertes inversiones en lo que ellos llamaron \emph{yield management} y lograron tener modelos de prónostico de demanda robustos, así como fuertes políticas de control de inventario y estrategías de sobreventa. Estas inversiones realizadas aunado al acto de desregularización de aerolíneas (1978) llevó a que \emph{American Airlines} tuviera tarifas muy competitivas disponibles para el mercado, promoviendo una sana competencia entre las diferentes aerolíneas de la época.
Luego del éxito de la implementación de las técnicas de \emph{yield management} en las aerolíneas, otras industrias comenzaron a implementar este tipo de ténicas. Una de estas industrias fue la hotelera, la cuál cuenta con una problemática similar a la presentada en las aerolíneas: inventario perecedro, clientes reservando cuartos con tiempo de anticipación, competencia ofreciendo tarifas de bajo costo, y grandes esfuerzos para poder balancear la oferta y la demanda. Sin embargo, el problema presentado en la industria de hoteles es más complejo aún ya que se debe tener en cuenta que un cliente puede permanecer en la propiedad más de una noche, afectando la oferta del hotel en días posteriores a las llegadas a sus huéspedes.
Las técnicas de \emph{revenue mangement / yield management} han sido estudiadas y explotadas de manera exitosa en la industria de la aerolínea, sin embargo hay mucho trabajo por hacer aún en otras industrias, incluyendo la hotelera.


\begin{itemize}
  \item Situación actual de RM
  \item Importancia del RM en la hotelería
  \item Componentes de RM: forecast de ocupación y modelo de pricing
  \item Técnicas más comúnes
  \item Objetivo de este trabajo
  \item Justificación de este trabajo
\end{itemize}

