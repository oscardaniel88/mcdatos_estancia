\chapter{Introducción}
\label{ch:intro}
Una práctica común dentro de la industria de la hotelería es la maximización del ingreso, mejor conocida como \emph{revenue management}. Esta surge de la necesidad de poder obtener vender cada uno de los cuartos dentro del inventario al mayor precio posible y de esta manera poder maximizar el ingreso percibido por la porpiedad derivado de la venta de cuartos (El-Gayar N, et al. 2008).
La técnica de \emph{revenue management} tiene como base el análisis de información generada por la misma propiedad así como información ajena a ella para poder vender el producto adecuado, al cliente adecuado, en el momento y precio adecuados. Estas estrategias ayudan al hotelero en su gestión diaria de diversas maneras. Le permite establecer una estrategia coherente de precios a todos los niveles, controlar la distribución multicanal, realizar previsiones de la demanda, conocer mejor las necesidades y patrones de reserva de los diversos segmentos de clientes y disponer de más información sobre el mercado y el entorno que lo envuelve.
Los sistemas de \emph{revenue management} aplican principios básicos de economía para configurar los precios y el control del inventario que se pretende vender, de hecho, hay tres categorías básicas de administración de demanda que son utilizadas dentro del \emph{revenue management}:
\begin{itemize}
  \item \emph{Decisiones Estructurales:} Se decide qué formato de venta se va a utilizar; a qué segmento de mercado se va a dirigir la venta; cuáles serán las condiciones de la venta que se ofrecerán.
  \item \emph{Decisiones de Precio:} Cómo se asignará el precio para los elementos dentro del inventario; diferentes categorías de precios en el mismo inventario;  cambios en los precios del inventario en el tiempo.
  \item \emph{Decisiones de Cantidad de Inventario:} Se decide si se acepta o se rechaza una oferta de compra; cuanto inventario se destina a cada uno de los segmentos de mercado; cuando retirar un producto del mercado para venderlo posteriormente.
\end{itemize}
Tradicionalmente los hoteles utilizan los controles de decisión de cantidad de inventario como su estrategía por \emph{"default"} (Aziz, et al. 2011), sin embargo la tecnología ha estado cambiando el mercado y la manera en que un hotel vende su inventario, ya que hoy en día los costos asociados al cambio de precio del inventario y el tiempo para distribuirlos ha bajado considerablemnte gracias a la adopción de un ecosistema totalmente electrónico (Bitran, Caldentey 2003).
Antes del auge del \emph{revenue management} la aerolínea \emph{British Airways} experimentaba aplicando promociones en distintos productos dentro de su inventario para generar demanda en los asientos que sin esta promoción hubieran volado vacíos. Posteriormente la dirección de la aerolínea \emph{American Airlines} dió un paso más adelante y crearon una practica llamada \emph{yield management}, la cual se enfoca principalmente en maximizar los ingresos de cada ruta vendida mediante un control delinventario basado en análisis de la información histórica disponible. \emph{American Airlines} hizo fuertes inversiones en lo que ellos llamaron \emph{yield management} y lograron tener modelos de prónostico de demanda robustos, así como fuertes políticas de control de inventario y estrategías de sobreventa. Estas inversiones realizadas aunado al acto de desregularización de aerolíneas (1978) llevó a que \emph{American Airlines} tuviera tarifas muy competitivas disponibles para el mercado, promoviendo una sana competencia entre las diferentes aerolíneas de la época.
Luego del éxito de la implementación de las técnicas de \emph{yield management} en las aerolíneas, otras industrias comenzaron a implementar este tipo de ténicas. Una de estas industrias fue la hotelera, la cuál cuenta con una problemática similar a la presentada en las aerolíneas: inventario perecedro, clientes reservando cuartos con tiempo de anticipación, competencia ofreciendo tarifas de bajo costo, y grandes esfuerzos para poder balancear la oferta y la demanda. Sin embargo, el problema presentado en la industria de hoteles es más complejo aún ya que se debe tener en cuenta que un cliente puede permanecer en la propiedad más de una noche, afectando la oferta del hotel en días posteriores a las llegadas a sus huéspedes.
Las técnicas de \emph{revenue mangement / yield management} han sido estudiadas y explotadas de manera exitosa en la industria de la aerolínea, sin embargo hay mucho trabajo por hacer aún en otras industrias, incluyendo la hotelera ya que en la medida en que esta técnica se afine los ingresos que este sector perciba por concepto de renta de habitaciones puede crecer sustancialmente.
Los sistemas existentes de \emph{revenue management} tienen dos componentes principales: un módulo que se encarga de pronósticar la ocupación del hotel en días futuros y un módulo que toma este pronóstico como base para recomendar precios para el inventario de tal forma que se maximice el ingreso total de la propiedad. Existen dos técnicas que son comúnmente utilizadas para hacer pronósticos de ocupación:
\begin{itemize}
  \item Análisis de índicadores históricos: Se construyen modelos de series de tiempo con los principales indicadores del hotel (habitaciones ocupadadas, reservaciones recibidas, etc) que ayudan a entender la temporalidad del hotel y el comportamiento de la demanda de esa propiedad en días espécificos. Este tipo de análisis suele tener errores muy altos \texttt{Weatherford (1998)}.
  \item Regresiones generalizadas, y modelos de \emph{machine learning}: En los casos en dónde se cuenta con información desagregada del hotel (información de cada una de las reservaciones) es posible aplicar modelos basados en regresiones generalizadas o redes neuronales, en las cuales se puede modelar el comportamiento de la demanda del hotel utilizando toda la información disponible de la propiedad reduciendo los errores de los pronósticos arrojados por el modelo \texttt{Caicedo-Torres, Payares (2016)}.
\end{itemize}
Es importante mencionar que una mejora del 10\% en la exactitud del pronóstico de la ocupación puede llevar a un incremento de entre el 0.5\% y 3.0\% en el ingreso del hotel (Weatherford, et al. 2003), es por ello que resulta de gran interés la mejora continua de los módelos de pronóstico de ocupación.
\subsection*{Objetivo}
El objetivo de el presente trabajo es presentar una propuesta de modelo de pronóstico de ocupación y maximización de ingresos para hoteles, mismo que pudiera ser encapsulado dentro de un sistema de \emph{revenue management}. El enfoque que se utilizó dentro de esta investigación para construír el modelo de pronóstico de ocupación fue la construcción de un modelo lineal generalizado (gracias a que se contaba con la información detallada de todas las reservaciones) cuyo resultado alimentaba a un segundo modelo que ofrece precios para tarifas y tipos de cuarto que maximizan el ingreso del hotel para una serie de fechas en el futuro.

Durante esta investigación se tuvo acceso a la base de datos de reservaciones para un hotel de negocios ubicado al sur de la Ciudad de México con información a partir del 01 de enero de 2015 hasta el 01 de agosto del 2018. También se utilizaron series de tiempo con el tipo de cambio del peso frente al dólar, una matriz de la calidad vs el precio de la propiedad y su set competitivo (definido como el conjunto de hoteles dentro de un radio de 5 km a la redonda y que compiten por la demanda del mismo mercado objetivo) y un calendario de eventos de la plaza en la cuál se encuentra ubicado el hotel.

La primera etapa de la investigación consistió en realizar un análisis exploratorio de los datos, donde se detectó el nivel de limpieza, y comportamiento de las distintas variables de interés. Una vez concluído el análisis se generó un modelo de pronóstico de ocupación, el cual genera curvas de \emph{pickup} de reservaciones, en las cuales se pueden observar los días de antelación con los que la propiedad vende los cuartos disponibles, y ajusta una regresión lineal generalizada con liga Poisson, el cual es un modelo usado comúnmente en problemas de conteos de datos y tablas de contingencia. La regresión lineal generalizada con liga Poisson asume que la variable de respuesta \emph{y} tiene una distribución de Poisson y también asume que el logaritmo de su valor esperado puede ser modelado por una comibinación lineal de los parámetros desconocidos. Una vez obtenido el pronóstico de ocupación se alimenta al modelo de maximización de ingresos, el cuál resuelve un problema de optimización con restricciones teniendo como función objetivo:$$\sum_{i=0}^{n}p_i*o_i$$

En donde i es la índice de la noche, $p_i$ es el precio del cuarto para la i-ésima noche y $o_i$ es la ocupación pronósticada (demanda) para la i-ésima noche al precio $p_i$

En este trabajo de investigación se presentarán los resultados de ambos modelos y los detalles sobre las limitantes así como los siguientes pasos para poder robustecer aún más este modelo.

\subsection*{Justificación}

Si bien los módelos de pronóstico de ocupación y los sistemas de \emph{revenue management} han madurado a lo largo de más de 30 años de haber sido implementados por primera vez en distintas industrias, aún hay mucho trabajo por hacer ya que el poder de cómputo y la capacidad de procesamiento ha aumentado significativamente a lo largo del tiempo abriendo nuevas oportunidades y facilitando la implementación de nuevas técnicas para el manejo de datos y el modelado estádistico que puede ser de gran ayuda durante la implementación de este tipo de modelos.

El mercado hotelero es altamente dinámico, genera inmensas cantidades de información día con día, misma que es almacenada en servidores y pocas veces es utilizada para poder obtener una ventaja competitiva. De ser exitoso el modelo aquí implementado se podría aprovechar la investigación realizada para generar un marco de trabajo que ayude a este sector en partícular a aprovechar la información generada detrás de sus datos, de tal suerte que puedan obtener mayores ingresos por los servicios ofrecidos generando un ambiente de competencia sana entre los distintos jugadores de este mercado.



