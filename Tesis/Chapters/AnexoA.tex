\chapter{Implementación de modelo de pronóstico de demanda utilizando un modelo lineal generalizado con liga Poisson}
\label{ch:anexoa}

\begin{verbatim}
library(plyr)
HIS_CEINS<-read_csv("datos/EST_PAS_UNIC_CEINS.csv")
RES_CEINS <- read.csv("datos/CEINS_RESERVAS.csv")
#Juntamos todas las tablas en un solo dataset
#datos <- rbind(HIS_CEINS, RES_CEINS)
datos <- HIS_CEINS
#Modelo de regresion Poisson para obtener
#parametros de modelo predictivo
modelo.1<-function(base,h){
  hotel<-subset(base,prop_code==h)
  dia<-sort(unique(hotel$date_in))
  ndia<-length(dia)
  beta0<-rep(0,ndia)
  beta1<-rep(0,ndia)
  variacion<-rep(0,ndia)
  for(i in 1:ndia){
    eleccion <- ddply(subset(hotel,as.character(date_in)==
                               as.character(dia[i]),
                             select = c('antelacion','nights')),
                      .(antelacion),summarise, 
                      nights = sum(nights))
    eleccion <- eleccion[with(eleccion,order(-antelacion)), ]
    eleccion$nights <- cumsum(eleccion$nights)
    eleccion <- eleccion[eleccion$antelacion<=60,]
    mod <- glm(nights~antelacion, family = "poisson", 
               data = eleccion)
    coeficientes <- coef(mod)
    beta0[i] <- coeficientes[1]
    beta1[i]<- coeficientes[2]
    variacion[i] <- sqrt(mod$deviance)
  }
  yresp <- data.frame(hotel=rep(h,ndia),dia,beta0,beta1,variacion)
  return(yresp)
}

# Prepara datos 2017 de entrada del modelo
res <- ddply(datos,.(prop_code,date_create,date_in), 
             summarise, nights=sum(nights))
res$prop_code<-toupper(res$prop_code)
res$antelacion <- as.numeric(res$date_in - res$date_create)
res <- subset(res,format(res$date_in, "%Y")=="2017")
yresp.1 <- modelo.1(res,"CEINS")
yresp<-yresp.1


### Preparacion de datos para modelo predictivo (2018)
series2018 <- subset(yresp,format(yresp$dia, "%Y")=="2017")
series2018$dia <- as.Date(series2018$dia)+365
names(series2018) <- c("hotel","dia","AAbeta0",
                       "AAbeta1","AAvariacion")

### Series a la fecha de corte en la extraccion

#Reservas
RES_CEXXX<-RES_CEINS
#
res<-RES_CEXXX
res <- ddply(res,.(prop_code,date_create,date_in), 
             summarise, nights=sum(nights))
res$prop_code<-toupper(res$prop_code)
res<-res %>% filter(date_create!="0000-00-00 00:00:00") %>% 
  mutate(date_create=as.Date(date_create),
  date_in=as.Date(date_in))
res$antelacion <- as.numeric(res$date_in - res$date_create)
series2018al11ago <- subset(series2018,as.numeric(dia)<
                              as.numeric(as.Date("2018-08-11")))
res2018al11ago <- subset(res,as.numeric(date_in)<
                           as.numeric(as.Date("2018-08-11")))
res2018al11ago <- subset(res2018al11ago,
                  format(res2018al11ago$date_in, "%Y")=="2018")

### Aplicacion del modelo a la fecha de corte por extraccion
param.1 <- modelo.1(res2018al11ago,"CEINS")

param2018 <- param.1
series2018al11ago <- join(series2018al11ago,param2018)

### Generacion de predictores de modelo predictivo
series2018al11ago$diasem <- as.factor(weekdays
                                      (series2018al11ago$dia))
series2018al11ago$mes <- as.factor(format(
  series2018al11ago$dia, "%b"))
series2018al11ago <- join(series2018al11ago,TDC)
series2018al11ago$eventos <- rep(0,nrow(series2018al11ago))
series2018al11ago$eventos <- as.factor(ifelse(
                               as.numeric(series2018al11ago$dia)
                             <=as.numeric(as.Date("2018-08-11")) & 
                               as.numeric(series2018al11ago$dia)>=
                               as.numeric(as.Date("2018-03-27")),
                               1,0))
#series2018al11ago <-join(series2018al11ago,eventos)
series2018al11ago$eventos<-as.factor(series2018al11ago$eventos)
PO_PT <- indicadores %>% select(Hotel,fecha,po,pt)

names(PO_PT) <- c("hotel","dia","PO","PT")
series2018al11ago <- join(series2018al11ago,PO_PT)

### Generacion de modelos predictivos por hotel
#CEINS
datos2018al11agoCEINS <- subset(series2018al11ago,hotel=="CEINS")
mod.CEINS.beta0 <- lm(beta0~AAbeta0+diasem+eventos+tdc+PO+PT,
                      data=datos2018al11agoCEINS)
datos2018al11agoCEINS$pred.beta0 <-predict(mod.CEINS.beta0,
                                           type='response')
datos2018al11agoCEINS$AAbeta1[is.na(
                             datos2018al11agoCEINS$AAbeta1)]<-0
mod.CEINS.beta1 <- lm(beta1~AAbeta1+diasem+eventos+tdc+PO+PT,
                      data=datos2018al11agoCEINS)
datos2018al11agoCEINS$pred.beta1<-predict(mod.CEINS.beta1,
                                          type='response')

### Aplica los modelos a la serie completa 2018
series2018 <- join(series2018,TDC)
series2018$AAbeta1[is.na(series2018$AAbeta1)]<-0
series2018 <- join(series2018,PO_PT)
series2018$diasem <- as.factor(weekdays(series2018$dia))
series2018$mes <- as.factor(format(series2018$dia, "%b"))

series2018$eventos1 <- ifelse(as.numeric(series2018$dia)<=
                                as.numeric(as.Date("2018-04-04")) & 
                                as.numeric(series2018$dia)>=
                                as.numeric(as.Date("2018-03-27")),
                                1,0)

series2018$eventos <- series2018$eventos1
series2018$eventos <- as.factor(series2018$eventos)
series2018$pred.beta0 <- ifelse(series2018$hotel=="CEINS",
                                predict(mod.CEINS.beta0,
                                        newdata=series2018,
                                        type="response"),0)
series2018$pred.beta1 <- ifelse(series2018$hotel=="CEINS",
                                predict(mod.CEINS.beta1,
                                        newdata=series2018,
                                        type="response"),0)
\end{verbatim}
