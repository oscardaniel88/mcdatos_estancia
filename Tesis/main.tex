% Tesis ITAM CLASS -- version 0.1 (13 - Abr - 2015)
% Clase para las tesis del ITAM
% 
% 13 - Abr - 2015 	Victor Martinez 	victor.martinez (at) itam.mx
% LICENSE: Creative Commons SA-BY 3.0
%
%
% Este documento presenta un ejemplo de uso de la plantilla
% El estudiante es libre de modificar este archivo a su gusto
% 



\documentclass{tesisITAM}

\usepackage[]{graphicx}\usepackage[]{color}
%% maxwidth is the original width if it is less than linewidth
%% otherwise use linewidth (to make sure the graphics do not exceed the margin)
\makeatletter
\def\maxwidth{ %
  \ifdim\Gin@nat@width>\linewidth
    \linewidth
  \else
    \Gin@nat@width
  \fi
}
\makeatother

\definecolor{fgcolor}{rgb}{0.345, 0.345, 0.345}
\newcommand{\hlnum}[1]{\textcolor[rgb]{0.686,0.059,0.569}{#1}}%
\newcommand{\hlstr}[1]{\textcolor[rgb]{0.192,0.494,0.8}{#1}}%
\newcommand{\hlcom}[1]{\textcolor[rgb]{0.678,0.584,0.686}{\textit{#1}}}%
\newcommand{\hlopt}[1]{\textcolor[rgb]{0,0,0}{#1}}%
\newcommand{\hlstd}[1]{\textcolor[rgb]{0.345,0.345,0.345}{#1}}%
\newcommand{\hlkwa}[1]{\textcolor[rgb]{0.161,0.373,0.58}{\textbf{#1}}}%
\newcommand{\hlkwb}[1]{\textcolor[rgb]{0.69,0.353,0.396}{#1}}%
\newcommand{\hlkwc}[1]{\textcolor[rgb]{0.333,0.667,0.333}{#1}}%
\newcommand{\hlkwd}[1]{\textcolor[rgb]{0.737,0.353,0.396}{\textbf{#1}}}%
\let\hlipl\hlkwb

\usepackage{float}
\usepackage{framed}
\makeatletter
\newenvironment{kframe}{%
 \def\at@end@of@kframe{}%
 \ifinner\ifhmode%
  \def\at@end@of@kframe{\end{minipage}}%
  \begin{minipage}{\columnwidth}%
 \fi\fi%
 \def\FrameCommand##1{\hskip\@totalleftmargin \hskip-\fboxsep
 \colorbox{shadecolor}{##1}\hskip-\fboxsep
     % There is no \\@totalrightmargin, so:
     \hskip-\linewidth \hskip-\@totalleftmargin \hskip\columnwidth}%
 \MakeFramed {\advance\hsize-\width
   \@totalleftmargin\z@ \linewidth\hsize
   \@setminipage}}%
 {\par\unskip\endMakeFramed%
 \at@end@of@kframe}
\makeatother

\definecolor{shadecolor}{rgb}{.97, .97, .97}
\definecolor{messagecolor}{rgb}{0, 0, 0}
\definecolor{warningcolor}{rgb}{1, 0, 1}
\definecolor{errorcolor}{rgb}{1, 0, 0}
\newenvironment{knitrout}{}{} % an empty environment to be redefined in TeX

\usepackage{alltt}
\usepackage[utf8]{inputenc}
\usepackage{lmodern}
\usepackage{amsmath}
\usepackage{enumitem}
\usepackage{graphicx}
\usepackage{listings}
\usepackage{booktabs}
\usepackage{longtable}
\usepackage{array}
\usepackage{multirow}
\usepackage[table]{xcolor}
\usepackage{wrapfig}
\usepackage{float}
\usepackage{colortbl}
\usepackage{pdflscape}
\usepackage{tabu}
\usepackage{threeparttable}
\usepackage{threeparttablex}
\usepackage[normalem]{ulem}
\usepackage{makecell}
\usepackage{csvsimple}
\usepackage[utf8]{inputenc}
\usepackage{listings}
\usepackage{tabularx}
\usepackage{ amssymb }

\lstset{commentstyle=\color{red},keywordstyle=\color{black},
showstringspaces=false}
\lstnewenvironment{rc}[1][]{\lstset{language=R}}{}
\newcommand{\ri}[1]{\lstinline{#1}}  %% Short for 'R inline'

\lstset{language=R} 

\title{Modelo de Predicción de Demanda y Maximización de Ingresos para Hoteles}
\author{Oscar Daniel Ezequiel Camarena Gómez}
\degree{Maestro en Ciencia de Datos}
\advisor{Mtro. José David Lampón Ortega}
\year{2019}

\begin{document}

	\pagenumbering{gobble}
	\maketitle
	\publicationrights
	\acknowledges

	%%%%%%%%%%%%%%%%%%%%%%%%%%%%%%%%%%%%%%%%%%%%%%
	% ABSTRACT
	%%%%%%%%%%%%%%%%%%%%%%%%%%%%%%%%%%%%%%%%%%%%%%

	%\begin{abstract}{spanish}
	
	%\end{abstract}


	\selectlanguage{spanish}
	\setcounter{page}{1}
	\pagenumbering{roman}

	\tableofcontents
	\listoffigures
	\listoftables
	\newpage

	\pagenumbering{arabic}
	\setcounter{page}{1}

	%%%%%%%%%%%%%%%%%%%%%%%%%%%%%%%%%%%%%%%%%%%%%%
	% CONTENT
	%%%%%%%%%%%%%%%%%%%%%%%%%%%%%%%%%%%%%%%%%%%%%%

	\chapter{Introducción}
\label{ch:intro}

\begin{itemize}
  \item Definir qué es Revenue Management
  \item Un poco de historia de RM
  \item Situación actual de RM
  \item Importancia del RM en la hotelería
  \item Componentes de RM: forecast de ocupación y modelo de pricing
  \item Técnicas más comúnes
  \item Objetivo de este trabajo
  \item Justificación de este trabajo
\end{itemize}



	\chapter{Marco teórico}
\label{ch:marco teorico}

En este capítulo se introducirá y discutirá el trabajo previo realizado y los conceptos necesarios para comprender las distintas técnicas y enfoques utilizados para la construcción de un módelo de predicción de demanda y maximización de ingresos así como el impacto que los resultados tienen dentro de la industria hotelera. Se dará una breve explicación del proceso de gestión de una propiedad mencionando cuales son los principales indicadores de control del proceso, así como las variables estudiadas para la toma de decisiones.

\section*{Gestion de una propiedad}

El proceso de gestión de propiedades o de hoteles comprende una serie de actividades que tienen como objetivo el garantizar la rentabilidad de un proyecto de arrendamiento. Dentro de las actividades que se llevan a cabo están las siguientes:
\begin{itemize}
  \item Gestión de \emph{canales de venta}
  \item Gestión de \emph{segmentos de mercado}
  \item Gestión de \emph{precios por tarifa}
  \item Análisis del \emph{comportamiento de la plaza}
  \item Análisis de \emph{indicadores principales}
\end{itemize}

A continuación profundizaremos en cada una de las actividades que forman parte de este proceso.

\subsection*{Gestión de canales de venta}

El equipo administrativo del hotel o propiedad, deben analizar diariamente el comportamiento de sus canales de venta o distribución. Típicamente un hotel tiene un catálogo estándar de canales de distribución:
\begin{itemize}
  \item Sitio Web 
  \item Call Center / FrontDesk
  \item Agencias de viajes en línea
  \item Globalizadores ~ Agencias de viajes
\end{itemize}

Cada uno de los canales de venta van dirigidos a un segmento de mercado en específico y lo que el equipo de administración debe hacer es asegurar que la demanda del hotel se mantenga lo suficientemente alta para poder asegurar la continuidad operativa del negocio. Para lograr esto se genera una estrategia de distribución, en la cuál se designa parte del inventario a cada uno de los canales y se fija un precio a cada una de las tarifas a ofertar. Los precios para un producto en específico tienden a variar entre los diferentes canales y esto tiene que ver con el costo asociado a la operación de cada uno de los canales. Por ejemplo, el sitio web típicamente es un canal de reservación propio del hotel, lo que significa que las ventas generadas por este canal son libres de comisiones; por el contrario, las agencias de viajes en línea tienden a cobrar una comisión por la venta generada por esos canales.

Lo que la administración del hotel debe conseguir es poder tener la mezcla de ventas entre sus canales que maximice el ingreso, de tal forma que los costos generados por las ventas no \emph{canibalicen} las ganancias de la propiedad. Es con esta restricción donde se descartaría una asignación total del inventario a las agencias de viajes en línea, ya que la comisión de este canal puede ser de hasta el 25\% del total de la venta, lo cual podría representar una pérdida de hasta $\frac{1}{4}$ de las ganancias que se pudieron haber obtenido. Si por el contrario asignaramos todo el inventario disponible a los canales propios del hotel, si bien no se paga una comisión, la demanda generada por estos canales es inferior a la demanda generada por las agencias de viajes en línea.

\subsection*{Gestión de segmentos de mercado}

La gestión de los segmentos de mercado es similar a la gestión de los canales de venta ya que estos estan relacionados a los canales de distribución. 

Las propiedades típicamente distinguen segmentan a sus clientes en las siguientes categorías:

\begin{itemize}
  \item \emph{Clientes directos:} Aquellos clientes que reservan en algún canal directo del hotel
  \item \emph{Clientes de negocios:} Aquellos clientes que pertenecen a una empresa que tiene alguna tarifa convenida con el hotel. Típicamente esta tarifa esta sujeta a una producción de cuartos durante un periodo
  \item \emph{Clientes de mayoreo:} Aquellos clientes que reservan utilizando un canal mayorista; puede ser una agencia de viajes en línea o mediante un agente de viajes.
  \item \emph{Grupos:} Aquellos clientes que llegan a la propiedad como parte de un grupo.
\end{itemize}

Para gestionar eficientemente los segmentos de mercado se debe crear una estrategia de asignación de precios a las diferentes tarifas que van dirigidas a cada uno de ellos. Por ejemplo, un \emph{cliente directo} que llega al hotel sin reservación, es propenso a pagar un precio mas alto por una habitación disponible, al contrario de un cliente que viene como parte de un grupo que reservó con mayor tiempo de anticipación con una tarifa mucho mas baja. El equipo administrativ\emph{}o debe cuidar los segmentos de mercado que ocupan la propiedad ya que el tener un grupo muy grande ocupándola significa muchas veces tener un hotel lleno a una tarifa muy baja y con mayor desgaste en las habitaciones, y por el otro lado, no siempre es posible llenar un hotel con clientes que llegan sin reservación los cuales están dispuestos a comprar una habitación a un precio mas elevado.

\subsection*{Gestión de precios por tarifa}

Una de las actividades mas importantes del proceso de gestión de una propiedad es la gestión de los precios por tarifa ya que generalmente las tarifas disponibles en un hotel están asociadas a un segmento de mercado y a un canal de reservación. Para poder realizar esta actividad se debe establecer una estrategia en la cual se define si se quiere incrementar la ocupación de la propiedad o si se quiere incrementar el \emph{RevPAR}(revenue per available room). Una vez establecida la estrategia inicial se deben modificar los precios para las tarifas que impactan los canales y segmentos donde se quiere fomentar la demanda o incrementar el precio. Si el objetivo es el primero, lo mas probable es que el cambio emplique una baja en las tarifas o creación de promociones sujetas a diversas restricciones, en cambio, si el objetivo es el segundo muy probablemente el cambio implique una alza en algunas tarifas y el cierre de algunas tarifas en donde este incremento no se pueda llevar a cabo por temas contractuales, por ejemplo las tarifas convenidas con otras empresas o clientes.

Es importante mencionar que la ejecución de esta tarea dentro del proceso de gestión depende del análisis continuo de la información histórica generada por el hotel (comportamiento de la propiedad en años anteriores en la misma fecha sujeta al estudio), conocimiento de la plaza en donde se ubica la propiedad (depende de la experiencia del equipo administrativo del hotel), información ajena al hotel (situación marco económica, eventos cercanos a la plaza, desempeño de la competencia cercana a la propiedad, etc). Muchas veces el conocer esta información implica un gran esfuerzo por parte del equipo ya que la información tiende a estar distribuída en diversos sistemas o en ocasiones deberán recabarla físicamente (visitando a su competencia o haciendo llamadas telefónicas) dejando un riesgo latente de recibir información que no es precisa.

Otra de las fuertes dependencias para la ejecución efectiva de esta tarea es contar con sistemas que distribuyan los nuevos precios en todos los canales disponibles en poco tiempo y con poco esfuerzo ya que de lo contrario no tendria sentido realizar esta labor si al final los precios no pueden ser cambiados dentro de un tiempo en el que el mercado pueda responder de acuerdo a lo planeado durante el análisis de la información. Una vez que los precios son alterados, el equipo deberá analizar nuevamente la información para asegurar que los cambios tuvieron los efectos deseados, de lo contrario, se deberá cambiar la estrategia lo antes posible para evitar afectar el desempeño de la propiedad.

\subsection*{Análisis del comportamiento de la plaza}

En el sector de la hotelería en México, se le llama plaza a la ubicación geográfica en dónde se encuentra una propiedad y típicamente agrupa a propiedades independientes o de otras marcas (set competitivo), oficinas, corporativos, centros comerciales, y lugares de interés turístico. La plaza genera gran cantidad de información que es de interés para el equipo administrativo de una propiedad, ya que el desempeño financiero de esta depende mucho de la situación que presenta la plaza en un momento determinado.

Los hoteles que comparten una plaza generalmente están dispuestos a compartir información de su propiedad con otros, de esa forma pueden tener un conocimiento mas amplio sobre la realidad que se vive en ella. Si una propiedad requiere obtener datos de su competencia, basta con una simple llamada telefónica a la propiedad de interés para solicitar los datos de interés, por ejemplo: \emph{cuartos ocupados}, \emph{cuartos disponibles}, \emph{cuartos fuera de servicio}, \emph{ingresos generados por la venta de habitaciones}. Con esta información se pueden calcular algunos indicadores que son indispensables para el an{alisis del comportamiento de la plaza:
\begin{itemize}
  \item \% de Ocupación: $\frac{cuartos\ ocupados}{cuartos\ disponibles}$
  \item ADR (Average Daily Rate o Tarifa Promedio): $\frac{ingresos}{cuartos\ ocupados}$
  \item RevPAR (Revenue per available room o Tarifa Efectiva): $\frac{ingresos}{cuartos\ disponibles}$
\end{itemize}

Para poder conocer la situación de una propiedad frente a la competencia en una plaza se generan otros indicadores que facilitan este tipo de análisis.

\begin{itemize}
  \item Penetración de ocupación: $\frac{\%\ ocupacion}{\%\ ocupacion\ plaza}$
  \item Penetración de tarifa: $\frac{tarifa\ promedio}{tarifa promedio\ plaza}$
\end{itemize}

El hotel o propiedad que cuente con los valores mas altos para estos indicadores será el hotel con mayor demanda y mayores ingresos dentro de la plaza, sin embargo, para que el análisis sea efectivo se debe escoger al set competitivo de manera adecuada, de tal forma que los hoteles contenidos dentro de este set sean de la misma gama y vayan dirigidos al mismo público objetivo.


\subsection*{Análisis de indicadores principales}

Como se mencionó anteriormente, existen indicadores de desempeño de una propiedad u hotel que deben ser estudiados de manera diaria para poder tomar decisiones que mejoren el desempeño de esta, asegurando una continuidad operativa y la rentabilidad del inmueble:

\begin{itemize}
  \item \% ocupación
  \item ADR
  \item REvpar
\end{itemize}

Típicamante durante el análisis realizado, se compara el desempeño de cada uno de estos indicadores de manera diaria, mensual y anual, comparando los valores contra el año anterior y buscando siempre una variación positiva entre años. En caso de tener variaciones negativas se debe buscar la causa de esta variación. Se puede tratar de una baja en la demanda general de la plaza, un periodo de recesión financiera generalizada, un nuevo competidor en la plaza, etc. Una vez identificado la causa se deben tomar las acciones necesarias para generar demanda a la propiedad en cuestión. Típicamente la forma más efectiva de hacerlo es modificar los precios de tal forma que el mercado responda favorablemente sin afectar la rentabilidad del inmueble.

\section*{Antecedentes: Modelos de Revenue Management}

Podemos definir la práctica de \emph{revenue management} como la aplicación de técnicas analíticas que intentan predecir el comportamiento de un consumidor al nivel de un micro mercado, optimizando la disponibilidad de productos y sus precios de tal forma que maximicen el crecimiento de los ingresos. El principal objetivo de esta práctica es vender el producto indicado al cliente indicado al tiempo indicado en el precio indicado. La escencia de esta práctica radica en entender la percepción del producto que tienen los clientes y alinear los precios de estos productos de acuerdo a esta haciendo una correcta segmentación de los clientes objetivos.

Es importante mencionar que la disciplina de \emph{revenue management} se compone de dos partes principales:
\begin{enumerate}
  \item Pronóstico de ocupación
  \item Asignación de precios
\end{enumerate}

En las siguientes secciones discutiremos acerca de los trabajos previos sobre los dos principales componentes de esta disciplina.

\subsection*{Trabajo previo: modelos de pronóstico de ocupación}

La base de una práctica efectiva de \emph{revenue management} es un buen modelo de pronóstico de demanda, ya que el resultado de este será la entrada del modelo de optimización de precio de tal suerte que mientras menor sea el error en el pronóstico de la demanda, mejor será la optimización de los precios para cada uno de los productos sujetos al análisis.

Se tiene registro de trabajo previo para el desarrollo de modelos de pronóstico de demanda en diferentes industrias, algunas de ellas son: \emph{líneas aéreas}, \emph{retail}, \emph{telecomunicaciones} y \emph{hospitalidad}, siendo la primera la que cuenta con modelos de pronóstico más maduros, este hecho obedece a que en esta industria es en dónde a principios de los años 80's nace la disciplina de \emph{revenue management}. 

La venta de asientos en un vuelo es un caso dónde naturalmente se puede utilizar un modelo de optimización de ingresos ya que cuenta con ciertas características que hacen que el modelado del problema sea muy intuitivo:
\begin{itemize}
  \item Se cuenta con un inventario reducido
  \item El inventario es perecedero (una vez que el avión despega, los asientos vacíos no generan un ingreso
  \item Se cuenta con un amplio segmento de mercado que genera demanda sobre el producto
  \item Los asientos se compran con distintos tiempos de anticipación (hay clientes que compran con un mayor tiempo de anticipación que otros)
  \item Se conoce el tiempo de estancia de cada uno de los clientes en la aeronave.
\end{itemize}

Las condiciones mencionadas anteriormente facilitan la construcción de un modelo de pronóstico de demanda. A continuación haremos una breve descripción de la evolución de este tipo de modelos a lo largo del tiempo.

\texttt{Beckmann y Bobkowski (1958)} presentaron modelos estadísticos que describen reservaciones de pasajeros, cancelación de reservaciones y \emph{no shows}. En su trabajo los autores comparan modelos utilizando diferentes distribuciones probabilísticas: \emph{Poisson}, \emph{Binomial Negativa} y \emph{Gamma} aplicadas en datos generados por aerolineas con la finalidad de modelar las reservaciones efectivas y propone una condicion optima para un nivel de sobreventa para un vuelo en partícular.

Posteriormente, otros autores continuaron explorando nuevas metodologías para poder construír modelos de predicción de demanda. Uno de los autores mas citados en las investigaciones de modelos de pronóstico de demanda es Lee (1990) que presenta un artículo en donde explica don enfoques para resolver el problema de pronóstico de ocupación en aerolíneas:
\begin{itemize}
  \item Modelos de series de tiempo 
  \item Modelos basados en información de reservaciones
\end{itemize}

Los modelos basados en series de tiempo consideran unicamente las series de tiempo con la información de llegadas o porcentajes de ocupación aplicandoles modelos de series de tiempo (como suavizamiento exponencial, ARIMA, etc). En estos casos no se utiliza la información propia de las reservaciones.

El enfoque basado en reservaciones hace uso de la información contenida en las reservaciones para pronosticar llegadas futuras. En este tipo de modelos típicamente se considera el concepto de \emph{pick-up}. Esto quiere decir que dadas \emph{K} reservaciones para un dia futuro \emph{T}, esperamos tener un \emph{pick-up} de \emph{N} reservaciones más desde este momento hasta el momento T. El pronóstico entonces será \emph{K + N}

Existen dos versiones del modelo de \emph{pick-up} (\texttt{Weatherford y Kimes}, 2003). La versión aditiva del modelo en donde a las reservaciones registradas en el hotel se le suman el promedio de reservaciones que típicamente llegan en un periodo entre la fecha actual y la fecha de llegada del huésped (tomando la temporalidad del hotel en cuenta), y la versión multiplicativa, la cual es muy similar, con la única diferencia de únicamente se toma en cuenta una fracción de las reservaciones actualmente registradas en el hotel.

Existen trabajos en los cuales se presenta un modelo avanzado en el cual se combinan modelos de series de tiempo y modelos basados en reservaciones, tal es el caso del artículo presentado por \texttt{Flides y Ord} (2002) en el cuál se concluye que la combinación de ambos modelos generalmente tienen un mejor desempeño en cuanto al pronóstico de la demanda. Otro trabajo que presenta un enfoque similar es el realizado por \texttt{Sa} (1987) en el cuál se utiliza una regresión multiple para desarrollar un modelo de pronóstico combinado, en este caso, la variable dependiente eran las reservaciones pendientes por llegar mientras la variable independiente incluía el número de resrvaciones confirmadas, un índice de temporalidad,  índice semanal y un promedio del comportamiento histórico de las reservaciones pendientes. Esta regresión se corría para varios días antes de la llegada de los clientes (t = 7, 14, 21, y 28).

Es importante hacer las siguientes observaciones sobre los modelos comentados anteriormente:

\begin{itemize}
  \item Los modelos de regesión lineal asumen que existe una correlación entre el número de reservaciones confirmadas y el número final de reservaciones (en el día 0), de tal forma que: $pronostico_{dia0} = a + b * reservaciones_{diaN}$
  \item Los modelos de regresión logarítmica asumen que existe una correlación entre el número de reservaciones confirmadas y el número final de reservaciones (en el día 0), de tal forma que: $log(pronostico_{dia0} = a + b * log(reservaciones_{diaN}$
  \item El modelo aditivo suma las reservaciones actuales al promedio histórico del \emph{pickup} de reservas desde el día de la lectura \emph{diaN} hasta el dia de la estancia \emph{dia0}: $pronostico_{dia0} = reservas_{diaN} + \sum_{i=0}^{n} \frac{pickup_i}{n}$
  \item El modelo multiplicativo multiplica las reservaciones actuales por la proporción del promedio histórico de reservaciones  desde el \emph{diaN} hasta el dia de la estancia \emph{dia0}: $pronostico_{dia0} = reservas_{diaN} * \frac{reservas_n - \sum_{i=0}^{n} \frac{pickup_i}{n}}{reservas_n}$
\end{itemize}

La tabla 2.1 resume los modelos más comunes utilizados para pronosticar ocupación, agrupados por categoría.


\begin{table}[]
\centering
\begin{tabular}{|c|l|}
\hline
\multicolumn{1}{|l|}{Categoria} & Modelo                                                                                                                                                    \\ \hline
Historico                       & \begin{tabular}[c]{@{}l@{}}1. Mismo día año anterior\\ 2. Promedios móviles\\ 3. Suvizamiento exponencial\\ 4. ARIMA\end{tabular}                         \\ \hline
Avanzado                        & \begin{tabular}[c]{@{}l@{}}1. Aditivo\\   a. Pickup Clasico\\   b. Pickup Avanzado\\ 2. Multiplicativo\\   a. Curva de reservaciones\end{tabular}         \\ \hline
Combinados                      & \begin{tabular}[c]{@{}l@{}}1. Promedios ponderados de modelo histórico y modelo avanzado\\ 2. Regresión \\ 3. Modelo de información completa\end{tabular} \\ \hline
\end{tabular}
\caption{Modelos de pronosótico por categoría}
\label{ModelosOcup}
\end{table}

\subsection*{Desempeño de modelos de pronóstico}

Se han realizado estudios con el objetivo de comparar el desempeño de los modelos previamente presentados, a continuación presentaremos los resultados obtenidos.

\texttt{Wickham} (1995) estudió la efectividad de una variedad de modelos de pronóstico de ocupación utilizando datos de aerolíneas. Durante este trabajo se comparan modelos históricos de ocupación (utilizando promedios simples y promedios ponderados) y modelos de \emph{pickup} (clasicos y avanzados) llegando a la conclusión de que los segundos son más efectivos, al menos para los datos aplicados durante su análisis.

\texttt{Weahterford} (1998) comparó los modelos aditivos frente a los modelos multiplicativos y modelos de regresión dentro del contexto de las aerolíneas y encontró que los modelos aditivos y de regresión tienen un mejor desempeño que los multiplicativos.

\subsection*{Modelos de machine learning}

Si bien las técnicas estadísticas anteriormente mencionadas pueden predecir efectivamente níveles de ocupación y demanda, se requieren de amplios conocimientos en estadística y de largos procedimientos para poderlas aplicar de manera tal que funcionen correctamente o de otra manera se puede optar por el uso de \emph{software} comercial el cuál resulta muy costoso la mayoría de las veces. Para solventar estos hechos, algunos autores presentan un enfoque diferente para construír modelos de predicción de ocupación, tal es el caso de \texttt{Caicedo-Torres, Payares} (2016) quienes proponen el uso de algorítmos de \emph{Machine Learning} para construír modelos predictivos de ocupación y demanda en la industria de la hospitalidad. La ventaja de estos modelos es que estan listos para ser utilizados por el staff administrativo de la propiedad  sin la necesidad de contar con amplios conocimientos de estadística. Además, estos modelos tienen la ventaja de poder ser empaquetados dentro de aplicaciones de bajo costo, o de ser ejecutados en una infraestructura basada en la nube, lo cual significan soluciones rentables para el sector de hospitalidad.

A continuación presentaremos a detalle algunos algorímtos de \texttt{M.L.} utilizados para la construcción de este tipo de modelos.

\subsubsection*{Regresión de Ridge}

La regresión de ridge es un algorítmo de regresión que incluye un término de regularización en su función de costos con el fin de lograr una mejor aproximación al momento de generalizar el modelo. La regresión de Ridge penaliza el algorítmo con un factor proporcional a la norma Euclidiana (L2) del vector de parámetros. $$J(\theta)=\sum_{i=1}^{n} (\theta^Tx_i - y_i)^2 + \alpha||\theta||^2$$

Dónde el factor de regularización $\alpha$ controla la importancia relativa de la penalidad compleja. Para optimizar esta ecuación se debe encontrar la derivada de la función de costos e igualarla a cero, de tal forma que el vector de parámetros que minimza el error es: $$\theta = (X^TX+\lambda I)^{-1} X^T y$$

Las nuevas predicciones se pueden calcular como sigue: $$h(X) = \theta^Tx=\sum_{i=1}^{n} \theta_ix_i$$

\subsubsection*{Ridge Regresión con Kernel}

Este método consta de aplicar el truco del Kernel a la regresión previamente comentada, esto con la finalidad de projectar los datos originales en un espectro de variables sin tener que explicitamente calcular las transformaciones. Para esto, se comienza con el vector de pesos optimizados para la Regresión de Ridge: $$\theta = (X^TX+\lambda I)^{-1} X^T y$$

Se manipula algebraícamente para obtener: $$\theta = X^T(XX^T + \lambda I)^{-1}y$$

De tal forma que una predicción puede ser obtenida por: $$h(x') = \theta^Tx' = y^T(XX^T + \lambda I)^{-1}Xx'$$

Una función de Kernel evalúa a un producto punto de la forma $$f(x_1,x_2) = \phi (x_1)\cdot \phi (x_2)$$ sin tener que explícitamente calcular la transformación $\phi$. Si definimos $K_{i,j}=f(x_i,x_j)$ y $k_i = f(x_i,x')$, donde $x_i$ es la iésima fila de la matriz $X$, entonces $$h(x') = \theta^Tx' = y^T(K + \lambda I)^{-1}k$$

Ajustamos un modelo en un espacio dimensional mayor inducido pro $\phi$ sin tener que incurrir en una complejidad computacional adicional. Se pueden utilizar varios Kernels, entre los más populares están el Kernel polinomial $$k(x_1,x_2)=( \gamma x_{1}^{T} x_2 + \beta)^ {\delta}$$

Y el Kernel Gaussiano $$k(x_1,x_2)=exp(-\frac{||x_1-x_2||^2}{2{\delta}^2})$$

\subsubsection*{Red neuronal multipcapa}

Una red neuronal es un modelo de \texttt{M.L.} capáz de clasificar y realizar regresión. Esta compuesta por $n$ capas totalmente conectadas de neuronas artificiales formando un grafo conectado (en el caso más simple se tiene una capa oculta y una capa de salida). En este modelo, la salida de cada capa de neuronas será la entrada de la siguiente capa. La función de activación es dada por la siguiente ecuación $$y=f(\sum_{i}^{n} w_ix_i+b)$$ donde $f$ es una función de activación. Entre las opciones más populares se encuentran la sigmoide $y = (1+exp(-x))^{-1}$ y la función lineal $y=x$. La elección de la función de activación en la capa de salida define el tipo de red neuronal (si clasifica o realiza regresión)

Las redes neuronales son entrenadas mediante un algorítmo llamado \emph{propagación hacia atrás (backpropagation)} que es una aplicación del algorítmo de descenso por gradiente que toma en cuenta la contribución de las neuronas de la capa oculta al error $E$

\subsubsection*{Red neuronal de base radial}

Las reded nueronales de base radial (RBF) usualmente tienen una sola capa oculta que emplea una serie de funciones llamadas \emph{funciones de base radial} como detectores de características. Las redes RBF pueden ser utilizadas para realizar regresiones o clasificaciones. En este modelo la capa oculta utiliza funcion gaussiana como función de activación $\phi$: $$\phi_j(x)=exp(-\frac{||x-\mu_j||^2}{2\sigma_j^2})$$

Y como función de salida, se tiene la siguiente ecuación: $$y(x)=\sum_{j=0}^{M} w_{kj} \phi_j(x)$$

donde $w_{kj}$ es el vector de pesos que conecta la neurona $j$ de la capa oculta con la neurona $k$ de la capa de salida.

El entrenamiento de este tipo de modelos consiste en encontrar un conjunto de valores para los parámetros $\mu_j$, $\sigma_j$ y $w_{jk}$ que maximizan el desempeño de la red. 

\subsection*{Desempeño de algoritmos de M.L.}



\subsection*{Modelos de Big Data}
\subsection*{Modelos de optimizacion de ingresos}
\subsection*{Resumen}

	\chapter{Metodología}
\label{ch:metodologia}

En el presente capítulo se presentará la metodología utilizada para la creación de un módulo propuesto para la predicción de demanda y maximización de ingresos en hoteles. La metodología guiará la estructura de este trabajo con el fin de hacer de él una contribución científica. La presente investigación se puede dividir en dos partes. La primera, consistió en el desarrollo un modelo capaz de predecir la demanda y ofrecer recomendaciones de precios para maximizar el ingreso diario de una propiedad. La segunda parte consistió en probar dicho modelo y discutir los resultados obtenidos con la finalidad de abrir nuevas líneas de investigación.

\section*{Diseño del modelo}

Para poder construír un modelo de predicción de demanda y maximización de ingresos para los hoteles que fueron sujetos al estudio, se siguieron una serie de pasos descritos a continuación:

\subsubsection*{Entendimiento del negocio}

El primer paso durante la construcción del modelo fue el entendimiento del negocio ya que para poder interpretar los datos es necesario comprender el negocio que está implicito en los datos. Durante esta etapa se trabajó con el equipo de gestión comercial de las propiedades para entender qué datos son analizados durante el proceso de toma de decisiones, de dónde son obtenidos estos datos y qué variables son de mayor interés para el negocio.

\subsubsection*{Identificación de las fuentes de datos}

Una vez comprendido el negocio y detectadas las variables de interés durante el proceso de la toma de decisiones, se identificaron las fuentes de datos, las cuales son nutridas por dos principales sistemas dentro de las propiedades: la central de reservaciones y el sistema de gestión de la propiedad. El primero es el sistema que sirve para dar servicio a todos los huéspedes que buscan una habitación mediante los canales de reservación disponibles (sitio web, call center, aplicaciones móviles, correos electrónicos, etc) y el segundo es que gestiona la propiedad y da servicio al huésped desde que llega hasta que se retira de la propiedad. Estos sistemas contienen toda la información del proceso desde que se encuentra en estatus de reservación hasta que llega a un estatus de \emph{check out}.

\subsubsection*{Análisis exploratorio de datos}

Luego de haber identificado las principales fuentes de datos, se diseñaron los procesos de extracción de información necesarios para obtener los datos que serían utilizados para la construcción del modelo. Para poder entender los datos, se ejecutó un análisis exploratorio de datos, en dónde se pudo observar el comportamiento de las variables de interés, su distribución probabilística, se detectaron temporalidades en los datos, nível de limpieza de datos y datos atípicos que pudieran generar problemas al momento de la implementación del modelo final.

\subsubsection*{Construcción del modelo}

	\chapter{Diseño y Desarrollo}
\label{ch:modelo}

Este capítulo describirá el diseño y desarrollo del modelo de predicción de demanda y optimización de ingresos. El primer paso será presentar los \emph{data sets} utilizados. El segundo paso consistirá en describir el modelo, sus módulos, el funcionamiento y la arquitectura.

\section*{Data Sets}

\subsection*{Detalle de reservaciones}

Para la construcción del modelo de predicción de demanda se utilizaron tres data sets que contienen la información suficiente para poder hacer un pronóstico efectivo de la demanda de una propiedad.

El primer data set utilizado fue extraído directamente del sistema de gestión de propiedades y contiene información de todas las reservaciones marcadas en los libros de la propiedad, así como la información de las estancias pasadas. A continuación se presenta a detalle la información contenida en este data set, el diccionario de datos y un breve resumen de la información en cada una de las columnas.

\subsubsection*{Diccionario de Datos}

A continuación se presenta el diccionario de datos del \emph{data set} que contiene la información detallada de las reservaciones:

\begin{itemize}
  \item rsrv\_code: Código de confirmación de la reservación
  \item date\_create: Fecha y hora de creación de la reserva
  \item date\_in: Fecha de llegada a la propiedad
  \item date\_out: Fecha de salida de la propiedad
  \item nights: Número de noches de la reservación en el hotel
  \item prop\_code: Código de la propiedad dentro del sistema de gestión de propiedades
  \item mkt\_sgm: Segmento de mercado del huésped amparado en la reservación
  \item Dia\_Sem: Día de la semana de la fecha de llegada del huésped
  \item rate\_code: Código de tarifa
  \item bucket: Categoría de tarifa
  \item Ingresos: Ingresos obtenidos por la renta habitación de la reservación
  \item rsrv\_src: Canal de generación de la reservación
  \item rsrv\_type: Tipo de reservación
  \item room\_type: Tipo de habitación
  \item PAX: Número de personas amparadas por la reservación
\end{itemize}

\subsubsection*{Resumen de datos}

\begin{verbatim}
##    rsrv_code        date_create                 
##  Min.   :6265739   Min.   :2016-07-20 18:31:59  
##  1st Qu.:7499788   1st Qu.:2017-03-27 10:10:09  
##  Median :7999769   Median :2017-06-30 12:12:57  
##  Mean   :8007253   Mean   :2017-06-27 13:38:02  
##  3rd Qu.:8505582   3rd Qu.:2017-09-27 13:22:36  
##  Max.   :9053354   Max.   :2018-01-01 03:08:21  
##     date_in                   
##  Min.   :2017-01-01 00:00:00  
##  1st Qu.:2017-04-04 00:00:00  
##  Median :2017-07-06 00:00:00  
##  Mean   :2017-07-06 02:18:17  
##  3rd Qu.:2017-10-08 00:00:00  
##  Max.   :2017-12-31 00:00:00  
##     date_out                       nights 
##  Min.   :2017-01-02 00:00:00   Min.   :1  
##  1st Qu.:2017-04-05 00:00:00   1st Qu.:1  
##  Median :2017-07-07 00:00:00   Median :1  
##  Mean   :2017-07-07 02:18:17   Mean   :1  
##  3rd Qu.:2017-10-09 00:00:00   3rd Qu.:1  
##  Max.   :2018-01-01 00:00:00   Max.   :1  
##   prop_code           mkt_sgm            Dia_Sem         
##  Length:44391       Length:44391       Length:44391      
##  Class :character   Class :character   Class :character  
##  Mode  :character   Mode  :character   Mode  :character  
##                                                          
##                                                          
##                                                          
##   rate_code            bucket             Ingresos   
##  Length:44391       Length:44391       Min.   :   0  
##  Class :character   Class :character   1st Qu.:1200  
##  Mode  :character   Mode  :character   Median :1462  
##                                        Mean   :1449  
##                                        3rd Qu.:1658  
##                                        Max.   :3900  
##    rsrv_src          rsrv_type          room_type        
##  Length:44391       Length:44391       Length:44391      
##  Class :character   Class :character   Class :character  
##  Mode  :character   Mode  :character   Mode  :character  
##                                                          
##                                                          
##                                                          
##       pax       
##  Min.   :0.000  
##  1st Qu.:1.000  
##  Median :1.000  
##  Mean   :1.271  
##  3rd Qu.:1.000  
##  Max.   :5.000
\end{verbatim}

Es importante resaltar que los datos fueron trabajados para presentarse de manera desagregada, es decir, si una reservación ampara una estancia de 5 noches, en el \emph{data set final} se presenta un registro por cada noche de la estancia, por ejemplo:

\begin{knitrout}
\definecolor{shadecolor}{rgb}{0.969, 0.969, 0.969}\color{fgcolor}\begin{table}[H]
\centering\rowcolors{2}{gray!6}{white}

\begin{tabular}{r|l|l|l|r|l}
\hiderowcolors
\hline
rsrv\_code & date\_create & date\_in & date\_out & nights & room\_type\\
\hline
\showrowcolors
6265739 & 2016-07-20 18:31:59 & 2017-04-07 & 2017-04-08 & 1 & NSK\\
\hline
6265739 & 2016-07-20 18:31:59 & 2017-04-08 & 2017-04-09 & 1 & NSK\\
\hline
6265739 & 2016-07-20 18:31:59 & 2017-04-09 & 2017-04-10 & 1 & NSK\\
\hline
6265739 & 2016-07-20 18:31:59 & 2017-04-10 & 2017-04-11 & 1 & NSK\\
\hline
6265739 & 2016-07-20 18:31:59 & 2017-04-11 & 2017-04-12 & 1 & NSK\\
\hline
6265739 & 2016-07-20 18:31:59 & 2017-04-12 & 2017-04-13 & 1 & NSK\\
\hline
\end{tabular}
\rowcolors{2}{white}{white}
\end{table}
\end{knitrout}

Como se puede observar, el número de reservación es el mismo en los seis registros, lo mismo ocurre con la fecha de creación, las noches y el tipo de habitación, sin embargo, la fecha de entrada y la fecha de salida va cambiando en cada registro.

\subsubsection*{Curvas de Pickup}

Del data set anterior podemos obtener las curvas de pickup del hotel. Estas curvas describen el comportamiento de las reservaciones para la propiedad en cuestión ya que mediante ellas podemos saber con cuanto tiempo de anticipación los huéspedes comienzan a reservar una habitación en la propiedad. A continuación se muestran las curvas de pickup para esta propiedad:

\definecolor{shadecolor}{rgb}{0.969, 0.969, 0.969}\color{fgcolor}
\includegraphics[width=\maxwidth]{Figures/pickup-1} 

\definecolor{shadecolor}{rgb}{0.969, 0.969, 0.969}\color{fgcolor}
\includegraphics[width=\maxwidth]{figures/pickupzoom-1} 

Analizando las gráficas presentadas anteriormente podemos concluír que la demanda de esta propiedad comienza 25 días antes de la llegada del huésped al hotel, e incrementa considerablemente 5 días antes de la llegada del huésped al hotel. Este dato nos indica que el huésped comienza a buscar una habitación dentro de esta propiedad en un lapso no mayor a 25 días antes de emprender su viaje.

Las curvas de \emph{pickup} son importantes dentro del proceso de toma de decisiones ya que le indican al equipo que gestiona la propiedad el tiempo de antelación con el cual deberían llevar a cabo las acciones necesarias para poder optimizar el ingreso generado por la renta de habitaciones.

\subsection*{Ocupación de la propiedad}

El segundo \emph{data set} con el que se trabajó contiene las líneas de tiempo de los níveles de ocupación de la propiedad. Este set de datos es de suma importancia ya que proporciona información relevante con respecto a las temporalidades del hotel, mismas que deben ser reflejadas en el resultado del modelo.

Las temporalidades del hotel pueden ser desde periodos con alta o baja ocupación, inclusive por día de la semana ya que al ser un hotel enfocado a viajes de negocio, se esperaría tener una alta ocupación los días laborales (Lunes a Jueves) y una baja ocupación los fines de semana (Viernes a Domingo).

A continuación se presentan un par de gráficas que describen el contenido del \emph{data set} en cuestión:

\definecolor{shadecolor}{rgb}{0.969, 0.969, 0.969}\color{fgcolor}
\includegraphics[width=\maxwidth]{figures/HabitacionesOcupadas-1} 

\definecolor{shadecolor}{rgb}{0.969, 0.969, 0.969}\color{fgcolor}
\includegraphics[width=\maxwidth]{Figures/Ocupacion-1} 

Como se puede observar en las gráficas presentadas anteriormente, esta propiedad tiene niveles de ocupación alrededor del 75\% llegando en repetidas ocasiones al 100\% de ocupación. Las caídas en los níveles de ocupación ocurren los fines de semana, lo cual confirma que el mercado objetivo de la propiedad estudiada es el turismo de negocios.

A continuación se presenta una gráfica de níveles de ocupación por día de la semana:

\definecolor{shadecolor}{rgb}{0.969, 0.969, 0.969}\color{fgcolor}
\includegraphics[width=\maxwidth]{Figures/Ocupacion_Dia_Semana-1}


\subsection*{Tarifa Promedio de la propiedad}

El tercer conjunto de datos utilizado contiene los valores de la tarifa promedio \emph{ADR: Average Daily Rate} en el tiempo. Estos datos fueron obtenidos directamente del PMS y es el resultado de sumar los ingresos generados por las habitaciones divididos por el número total de cuartos vendidos. 

A continuación se presenta la línea de tiempo con los valores para la tarifa promedio a lo largo del tiempo.

\definecolor{shadecolor}{rgb}{0.969, 0.969, 0.969}\color{fgcolor}
\includegraphics[width=\maxwidth]{figures/IndicadoresTarifaPromedio-1} 

De esta última gráfica se puede observar que los precios siguen una tendencia positiva con respecto al tiempo, a medida que pasa el tiempo, los precios incrementan, sin embargo podemos notar también que a medida que pasa el tiempo los precios varían más. Si contrastamos la información presentada en la gráfica de la tarifa promedo vs el tiempo y la presentada en la gráfica de ocupación vs el tiempo podemos ver que aunque los precios han aumentado en esta propiedad, las tendencias en los níveles de ocupación permanecen constantes.

\subsection*{Precios Públicos en el tiempo}

Como se mencionó anteriormente, una propiedad típicamente tiene diferentes precios disponibles para cada uno de los segmentos de mercado a los que atiende. La tarifa que se ofrece al público que llega al hotel sin una reservación previa se le conoce como \emph{tarifa pública} y generalmente es la tarifa con el precio mas alto de la cúal desprenden todas las demás tarifas, es decir, el conjunto de tarifas disponible en el hotel serán descuentos realizados sobre la \emph{tarifa pública}.

El tercer data set con el se trabajo el desarrollo del modelo contiene la información de las tarifas públicas para el hotel en cuestión y su competencia a lo largo del tiempo. Esto nos ayudará a conocer el precio del producto ofrecido en el mercado objetivo, de tal forma que los precios arrojados por el modelo se encuentren siempre dentro del mercado.

Los precios de la competencia fueron obtenidos de las páginas electrónicas de las agencias en línea, booking.com y expedia.com principalmente.

A continuación se presentan las líneas de tiempo con los precios públicos para el conjunto de propiedades.

\definecolor{shadecolor}{rgb}{0.969, 0.969, 0.969}\color{fgcolor}
\includegraphics[width=\maxwidth]{figures/PreciosGraf-1} 

En la gráfica anterior se puede observar que hay algunos datos atípicos ya que es poco probable encontrar habitaciones en este segmento de hoteles por arriba de los \$3,000 MXN. Para confirmar esto se realizó una gráfica de caja y bigotes (boxplot) en la cuál se puede observar a primera vista los datos atípicos contenidos en el \emph{data set}.

\definecolor{shadecolor}{rgb}{0.969, 0.969, 0.969}\color{fgcolor}
\includegraphics[width=\maxwidth]{figures/PreciosBoxPlot-1} 

Los datos atípicos fueron eliminados antes de proceder a trabajar con el módelo final.

\subsection*{Tipo de cambio}

El último \emph{data set} utilizado para la construcción del modelo contiene la información histórica de los tipos de cambio del peso frente al dólar.

\definecolor{shadecolor}{rgb}{0.969, 0.969, 0.969}\color{fgcolor}
\includegraphics[width=\maxwidth]{figures/TiposdeCambio-1} 

En el siguiente apartado detallaremos cómo se utilizaron los datos presentados para la construcción del modelo de predicción de demanda y maximización de ingresos.


\section*{Modelo de Predicción de Demanda}

El modelo de predicción de demanda toma como entrada los datos de las curvas de \emph{pickup} para el hotel para cada día del periodo estudiado. Estas curvas describen la velocidad con la que una propiedad vende su inventario para un día en específico de tal forma que el modelo propuesto pueda ajustar una curva que resuma el comportamiento de la propiedad en un día similar al estudiado y de esta forma se pueda generar unna predicción de los cuartos vendidos en la propiedad y los días de antelación con los que serán vendidos.

Este modelo está compuesto por dos módulos principales. El primero es una regresión \emph{Poisson} definida como $log(E(Y|X)) = \beta_0 + \beta_1X$, que se alimentará con las curvas de \emph{pickup} que describen el comportamiento del hotel. Cómo resultado de este módulo obtendremos un parámetro $\beta_0$ y un parámetro $\beta_1$ que describen la curva que mejor ajusta al \emph{pickup} del hotel.

A continuación se muestra un resumen del conjunto de datos que alimentan al primer módulo del modelo de predicción de demanda:

\begin{verbatim}
 prop_code date_create    date_in nights antelacion
1      CEINS  2017-03-25 2017-03-26      1          1
2      CEINS  2016-12-26 2017-01-01      3          6
3      CEINS  2016-12-26 2017-01-02      3          7
4      CEINS  2016-12-26 2017-01-05      1         10
5      CEINS  2016-12-26 2017-01-03      1          8
6      CEINS  2016-12-26 2017-01-04      1          9
7      CEINS  2016-12-26 2017-01-05      1         10
8      CEINS  2016-12-26 2017-01-01      1          6
9      CEINS  2016-12-26 2017-01-02      1          7
10     CEINS  2016-12-26 2017-01-03      1          8
11     CEINS  2016-12-26 2017-01-04      1          9
12     CEINS  2016-12-27 2017-01-02      1          6
13     CEINS  2016-12-27 2017-01-03      1          7
14     CEINS  2016-12-27 2017-01-03      1          7
15     CEINS  2016-12-27 2017-01-04      1          8
\end{verbatim}

Una vez obtenidos los parámetros $\beta_0$ y $\beta_1$ de la regresión \emph{Poisson}, se procede a generar predictores potenciales de la demanda a partir de los datos con los que contamos para este análisis.

Los predictores definidos son:

\begin{itemize}
  \item diaSem: Definido como factores que van de Lunes a Domingo
  \item Mes: Definido como factores que van de Enero a Diciembre
  \item Eventos: 1 si la plaza cuenta con algun evento social, deportivo, o de entretenimiento en esa fecha, 0 en caso contrario
  \item PT: Definido como la tarifa promedio de la propiedad dividida entre la tarifa promedio de la competencia
  \item TDC: El valor del peso frente al dólar para cada día
  \item $beta_0$: El parámetro $\beta_0$ obtenido de la regresión \emph{Poisson}
  \item $beta_1$: El parámetro $\beta_1$ obtenido de la regresión \emph{Poisson}
\end{itemize}







	\chapter{Interpretación y Resultados}
\label{ch:results}

En el presente capítulo se presentarán a detalle los resultados obtenidos por el módelo de pronóstico de demanda y maximización de ingresos. También se discutirá la forma en la que se deberán interpretar los resultados del mismo y así evitar sacar conclusiones erróneas o fuera de contexto.

\section*{Interpretación del modelo}


	
	\chapter{Conclusiones}
\label{ch:conclusiones}

En este trabajo se compararon distinas técnicas con miras a identificar cuál de ellas ofrece mejores resultados en la predicción de demanda de habitaciones y níveles de ocupación de propiedades en el sector hotelero.

Las técnicas comparadas incluyeron distintos enfoques: desde enfoques tradicionales (regresión lineal generalizada con liga Poisson), aprendizaje de máquina (regresión de Ridge) y análisis de series de tiempo (SARIMA). 

Los resultados encontrados muestran la superioridad de la téncina SARIMA en el pronóstico de la ocupación con un error \emph{MAPE} encontrado de 14.8880 sobre el conjunto de pruebas. Las predicciones obtenidas por el modelo mostraron ser buenas aproximaciones y no se encontró evidencia de sobreajuste del modelo.

Adicionalmente se ejecutó un ejercicio que muestra una aplicación de un modelo de predicción dentro de un proceso de toma de decisiones. Para ello se implentó un modelo cuyo objetivo es maximizar los ingresos de una propiedad dado un pronóstico de ocupación en días futuros. Esta parte de la investigación mostró que las decisiones que se tomen dentro de una empresa o cualquier negocio pueden ser optimizadas si están basadas en el análisis de la información que se tenga disponible. El modelo de optimización de ingresos presentado en este trabajo muestra el beneficio que podría obtener la propiedad si fijaran los precios basándose en los resultados obtenidos por el modelo de pronóstico de demanda.

En general, en nuestro país no hemos llegado a explotar las bondades que ofrece el análisis de datos y las técnicas de inteligancia artificial en el campo hotelero. Con los avances en la tecnología, la alta capacidad de cómputo y los lenguajes especializados de programación, como por ejemplo \emph{Python} que ofrecen librerías de uso libre con algorítmos implementados de aprendizaje de máquina y de técnicas de análisis de información, es mas asequible la aplicación de este tipo de técnicas, logrando con esto soluciones que puedan impactar de manera sustancial la competitividad en la industria hotelera.

Para trabajos futuros se sugiere incluír en los experimentos otras variables independientes asociadas con eventos y días especiales, información del tráfico generado en el sitio web del hotel, número de llamadas a recepción o a la central de reservas, con el objeto de evaluar si estas variables son representativas y pueden contribuír en la minimización del error de validación atrapando parte del patrón generador de la ocupación diaria, referente al viajero de negocios y al de placer.

Otro aspecto importante sugerido para trabajos futuros es robustecer el modelo de optimización de ingresos, ya que para fines de este trabajo se implementó tomando en cuenta algunas consideraciones que limitan su desempeño.



	

	%%%%%%%%%%%%%%%%%%%%%%%%%%%%%%%%%%%%%%%%%%%%%%
	% APPENDIX
	%%%%%%%%%%%%%%%%%%%%%%%%%%%%%%%%%%%%%%%%%%%%%%
	\appendix
	% \include{Chapters/appendixA}
	\chapter{Implementación de modelo de pronóstico de demanda utilizando un modelo lineal generalizado con liga Poisson}
\label{ch:anexoa}

\begin{verbatim}
library(plyr)
HIS_CEINS<-read_csv("datos/EST_PAS_UNIC_CEINS.csv")
RES_CEINS <- read.csv("datos/CEINS_RESERVAS.csv")
#Juntamos todas las tablas en un solo dataset
#datos <- rbind(HIS_CEINS, RES_CEINS)
datos <- HIS_CEINS
#Modelo de regresion Poisson para obtener
#parametros de modelo predictivo
modelo.1<-function(base,h){
  hotel<-subset(base,prop_code==h)
  dia<-sort(unique(hotel$date_in))
  ndia<-length(dia)
  beta0<-rep(0,ndia)
  beta1<-rep(0,ndia)
  variacion<-rep(0,ndia)
  for(i in 1:ndia){
    eleccion <- ddply(subset(hotel,as.character(date_in)==
                               as.character(dia[i]),
                             select = c('antelacion','nights')),
                      .(antelacion),summarise, 
                      nights = sum(nights))
    eleccion <- eleccion[with(eleccion,order(-antelacion)), ]
    eleccion$nights <- cumsum(eleccion$nights)
    eleccion <- eleccion[eleccion$antelacion<=60,]
    mod <- glm(nights~antelacion, family = "poisson", 
               data = eleccion)
    coeficientes <- coef(mod)
    beta0[i] <- coeficientes[1]
    beta1[i]<- coeficientes[2]
    variacion[i] <- sqrt(mod$deviance)
  }
  yresp <- data.frame(hotel=rep(h,ndia),dia,beta0,beta1,variacion)
  return(yresp)
}

# Prepara datos 2017 de entrada del modelo
res <- ddply(datos,.(prop_code,date_create,date_in), 
             summarise, nights=sum(nights))
res$prop_code<-toupper(res$prop_code)
res$antelacion <- as.numeric(res$date_in - res$date_create)
res <- subset(res,format(res$date_in, "%Y")=="2017")
yresp.1 <- modelo.1(res,"CEINS")
yresp<-yresp.1


### Preparacion de datos para modelo predictivo (2018)
series2018 <- subset(yresp,format(yresp$dia, "%Y")=="2017")
series2018$dia <- as.Date(series2018$dia)+365
names(series2018) <- c("hotel","dia","AAbeta0",
                       "AAbeta1","AAvariacion")

### Series a la fecha de corte en la extraccion

#Reservas
RES_CEXXX<-RES_CEINS
#
res<-RES_CEXXX
res <- ddply(res,.(prop_code,date_create,date_in), 
             summarise, nights=sum(nights))
res$prop_code<-toupper(res$prop_code)
res<-res %>% filter(date_create!="0000-00-00 00:00:00") %>% 
  mutate(date_create=as.Date(date_create),
  date_in=as.Date(date_in))
res$antelacion <- as.numeric(res$date_in - res$date_create)
series2018al11ago <- subset(series2018,as.numeric(dia)<
                              as.numeric(as.Date("2018-08-11")))
res2018al11ago <- subset(res,as.numeric(date_in)<
                           as.numeric(as.Date("2018-08-11")))
res2018al11ago <- subset(res2018al11ago,
                  format(res2018al11ago$date_in, "%Y")=="2018")

### Aplicacion del modelo a la fecha de corte por extraccion
param.1 <- modelo.1(res2018al11ago,"CEINS")

param2018 <- param.1
series2018al11ago <- join(series2018al11ago,param2018)

### Generacion de predictores de modelo predictivo
series2018al11ago$diasem <- as.factor(weekdays
                                      (series2018al11ago$dia))
series2018al11ago$mes <- as.factor(format(
  series2018al11ago$dia, "%b"))
series2018al11ago <- join(series2018al11ago,TDC)
series2018al11ago$eventos <- rep(0,nrow(series2018al11ago))
series2018al11ago$eventos <- as.factor(ifelse(
                               as.numeric(series2018al11ago$dia)
                             <=as.numeric(as.Date("2018-08-11")) & 
                               as.numeric(series2018al11ago$dia)>=
                               as.numeric(as.Date("2018-03-27")),
                               1,0))
#series2018al11ago <-join(series2018al11ago,eventos)
series2018al11ago$eventos<-as.factor(series2018al11ago$eventos)
PO_PT <- indicadores %>% select(Hotel,fecha,po,pt)

names(PO_PT) <- c("hotel","dia","PO","PT")
series2018al11ago <- join(series2018al11ago,PO_PT)

### Generacion de modelos predictivos por hotel
#CEINS
datos2018al11agoCEINS <- subset(series2018al11ago,hotel=="CEINS")
mod.CEINS.beta0 <- lm(beta0~AAbeta0+diasem+eventos+tdc+PO+PT,
                      data=datos2018al11agoCEINS)
datos2018al11agoCEINS$pred.beta0 <-predict(mod.CEINS.beta0,
                                           type='response')
datos2018al11agoCEINS$AAbeta1[is.na(
                             datos2018al11agoCEINS$AAbeta1)]<-0
mod.CEINS.beta1 <- lm(beta1~AAbeta1+diasem+eventos+tdc+PO+PT,
                      data=datos2018al11agoCEINS)
datos2018al11agoCEINS$pred.beta1<-predict(mod.CEINS.beta1,
                                          type='response')

### Aplica los modelos a la serie completa 2018
series2018 <- join(series2018,TDC)
series2018$AAbeta1[is.na(series2018$AAbeta1)]<-0
series2018 <- join(series2018,PO_PT)
series2018$diasem <- as.factor(weekdays(series2018$dia))
series2018$mes <- as.factor(format(series2018$dia, "%b"))

series2018$eventos1 <- ifelse(as.numeric(series2018$dia)<=
                                as.numeric(as.Date("2018-04-04")) & 
                                as.numeric(series2018$dia)>=
                                as.numeric(as.Date("2018-03-27")),
                                1,0)

series2018$eventos <- series2018$eventos1
series2018$eventos <- as.factor(series2018$eventos)
series2018$pred.beta0 <- ifelse(series2018$hotel=="CEINS",
                                predict(mod.CEINS.beta0,
                                        newdata=series2018,
                                        type="response"),0)
series2018$pred.beta1 <- ifelse(series2018$hotel=="CEINS",
                                predict(mod.CEINS.beta1,
                                        newdata=series2018,
                                        type="response"),0)
\end{verbatim}

	
	\chapter{Anexo B: Implementación de modelo de pronóstico de demanda utilizando una regresión de Ridge}
\label{ch:anexob}

\begin{verbatim}
import pandas as pd
import numpy as np
import random
import matplotlib.pyplot as plt
%matplotlib inline
from matplotlib.pylab import rcParams
rcParams['figure.figsize'] = 12, 10

indicadores = pd.read_excel("data/indicadores_CEINS.xls")
indicadores = indicadores[['fecha','occ']]

def regresa_ds(data,window=1,dropnan=True):
    for x in range(1,window+1):
        column = 't-'+str(window-x)
        data[column]=data['occ'].shift(window-x)
    if dropnan:
        data.dropna(inplace=True)
    data=data.drop('occ',1)
    return data
    
lags = 400
indicadores = regresa_ds(indicadores,lags,1)

import datetime
indicadores['fecha']=pd.to_datetime(indicadores['fecha'],format='%Y-%m-%d')
indicadores['dow']=indicadores['fecha'].dt.strftime('%A')
indicadores['year']=indicadores['fecha'].dt.strftime('%Y')
indicadores['month']=indicadores['fecha'].dt.strftime('%m')
indicadores['day']=indicadores['fecha'].dt.strftime('%d')
indicadores['y']=indicadores['t-0']
df_dow = pd.get_dummies(indicadores['dow'])
indicadores = pd.concat([indicadores,df_dow],axis=1)
indicadores=indicadores.drop('dow',1)
indicadores=indicadores.drop('t-0',1)
indicadores.to_csv('data/indicadores.csv', sep=',')

train_size = int(len(indicadores)*0.66)
train, test = indicadores[0:train_size], indicadores[train_size:len(indicadores)]

from sklearn.linear_model import Ridge
def ridge_regression(data, predictors, alpha, models_to_plot={}):
    #Fit the model
    ridgereg = Ridge(alpha=alpha,normalize=True)
    ridgereg.fit(data[predictors],data['y'])
    y_pred = ridgereg.predict(data[predictors])
    
    #Check if a plot is to be made for the entered alpha
    if alpha in models_to_plot:
         plt.subplot(models_to_plot[alpha])
         plt.tight_layout()
         plt.plot(data['fecha'],y_pred)
         plt.plot(data['fecha'],data['y'],'.')
         plt.title('Plot for alpha: %.3g'%alpha)
    
    #Return the result in pre-defined format
    rss = sum((y_pred-data['y'])**2)
    ret = [rss]
    ret.extend([ridgereg.intercept_])
    ret.extend(ridgereg.coef_)
    return ret, y_pred
    
predictors=['t-1']
predictors.extend(['t-%d'%i for i in range(2,lags)])
predictors = predictors + 
['year','month','day','Monday','Tuesday','Wednesday',
'Thursday','Friday','Saturday','Sunday']
alpha_ridge = [1e-15, 1e-10, 1e-8, 1e-4, 1e-3,1e-2, .1, 5, 10, 20]

col = ['rss','intercept'] + ['coef_x_%d'%i for i in range(1,len(predictors)+1)]
ind = ['alpha_%.2g'%alpha_ridge[i] for i in range(0,10)]
coef_matrix_ridge = pd.DataFrame(index=ind, columns=col)
y_pred = [None] * 10
models_to_plot = {1e-15:231, 1e-10:232, 1e-4:233, 1e-3:234, .1:235, 5:236}
for i in range(10):
    coef_matrix_ridge.iloc[i,],y_pred[i] = ridge_regression(train, 
    predictors, alpha_ridge[i], models_to_plot)
    
alpha = 1e-15
predictors=['t-1']
predictors.extend(['t-%d'%i for i in range(2,lags)])
predictors = predictors + ['year','month','day','Monday','Tuesday','Wednesday','Thursday','Friday','Saturday','Sunday']
ridgereg = Ridge(alpha=alpha,normalize=True)
ridgereg.fit(train[predictors],train['y'])
y_pred_test = ridgereg.predict(test[predictors])
MAPE = (sum(abs((test['y']-y_pred_test)/(test['y'])))/len(test['y'])) * 100
print(MAPE)

import datetime
from datetime import timedelta  
pronostico_raw = pd.read_csv("data/Pronostico_CEINS.csv")
pronostico_raw = pronostico_raw[['fecha','occ']]
pronostico_raw['fecha']=pd.to_datetime(pronostico_raw['fecha'],format='%Y-%m-%d')

fecha_pron =  datetime.date(2018, 8,14)
fecha_inicio = fecha_pron

predictors=['t-1']
predictors.extend(['t-%d'%i for i in range(2,lags)])
predictors = predictors + ['year','month','day','Monday',
'Tuesday','Wednesday','Thursday','Friday','Saturday','Sunday']

window=40
i=0


while i < window:
    pronostico = pronostico_raw.copy()
    pronostico = regresa_ds(pronostico,lags,1)
    pronostico['fecha']=pd.to_datetime(pronostico['fecha'],format='%Y-%m-%d')
    pronostico['dow']=pronostico['fecha'].dt.strftime('%A')
    pronostico['year']=pronostico['fecha'].dt.strftime('%Y')
    pronostico['month']=pronostico['fecha'].dt.strftime('%m')
    pronostico['day']=pronostico['fecha'].dt.strftime('%d')
    pronostico['y']=pronostico['t-0']
    df_dow = pd.get_dummies(pronostico['dow'])
    pronostico = pd.concat([pronostico,df_dow],axis=1)
    pronostico=pronostico.drop('dow',1)
    pronostico=pronostico.drop('t-0',1)
    pronostico = pronostico[(pronostico.fecha >=fecha_pron)]
    aux=pronostico.loc[pronostico['fecha'] == fecha_pron]
    aux=aux[predictors]
    y_pred_test = ridgereg.predict(aux)
    pronostico_raw.loc[pronostico_raw['fecha'] == 
    fecha_pron,'occ']=y_pred_test
    fecha_pron=fecha_pron+timedelta(days=1)
    i = i+1
    

fecha_fin = fecha_pron-timedelta(days=1)


pronostico_raw = pronostico_raw[(pronostico_raw.fecha >=fecha_inicio)&
(pronostico_raw.fecha <=fecha_fin)]     
pronostico_raw['Hab_Ocup']=round(pronostico_raw['occ']*159)
pronostico_raw.to_csv('data/pronostico.csv', sep=',')



\end{verbatim}

	
	\chapter{Anexo C: Implementación de modelo de pronóstico de demanda utilizando un modelo de análisis de series de tiempo ARIMA}
\label{ch:anexoa¡c}

\begin{verbatim}
import warnings
import itertools
import numpy as np
import pandas as pd
import statsmodels.api as sm
import matplotlib.pyplot as plt
%matplotlib inline
from matplotlib.pylab import rcParams
rcParams['figure.figsize'] = 12, 10

df = pd.read_csv("data/Indicadores_CEINS.csv")
ocupacion = df.loc[df['fecha'] >= '2013-01-01']
ocupacion['fecha'].min(), ocupacion['fecha'].max()
ocupacion.head()

cols = ['Hotel','habs_disp','tp','po','pt','te']
ocupacion = ocupacion.drop(cols, axis=1)
ocupacion = ocupacion.sort_values('fecha')
ocupacion.isnull().sum()
ocupacion.head()

ocupacion = ocupacion.groupby('fecha')['occ'].sum().reset_index()

ocupacion['fecha']=pd.to_datetime(ocupacion['fecha'])
ocupacion = ocupacion.set_index('fecha')

y = ocupacion

p = d = q = range(0, 2)
pdq = list(itertools.product(p, d, q))
seasonal_pdq = [(x[0], x[1], x[2], 60) for x in list(itertools.product(p, d, q))]
print('Examples of parameter combinations for Seasonal ARIMA...')
print('SARIMAX: {} x {}'.format(pdq[1], seasonal_pdq[1]))
print('SARIMAX: {} x {}'.format(pdq[1], seasonal_pdq[2]))
print('SARIMAX: {} x {}'.format(pdq[2], seasonal_pdq[3]))
print('SARIMAX: {} x {}'.format(pdq[2], seasonal_pdq[4]))

for param in pdq:
    for param_seasonal in seasonal_pdq:
        try:
            mod = sm.tsa.statespace.SARIMAX(y,order=param,
            seasonal_order=param_seasonal,
            enforce_stationarity=False,
            enforce_invertibility=False)
            results = mod.fit()
            print('ARIMA{}x{}60 - AIC:{}'.format(param, 
            param_seasonal, 
            results.aic))
        except:
            continue
            
mod = sm.tsa.statespace.SARIMAX(y,
                                order=(1, 1, 1),
                                seasonal_order=(1, 0, 1, 60),
                                enforce_stationarity=False,
                                enforce_invertibility=False)
results = mod.fit()
print(results.summary().tables[1])

pred = results.get_prediction(start=pd.to_datetime('2017-01-01'), dynamic=False)
pred_ci = pred.conf_int()
ax = y['2014':].plot(label='observed')
pred.predicted_mean.plot(ax=ax, 
label='Pronóstico un paso adelante', 
alpha=.7, figsize=(14, 7))
ax.fill_between(pred_ci.index,
                pred_ci.iloc[:, 0],
                pred_ci.iloc[:, 1], color='k', alpha=.2)
ax.set_xlabel('Date')
ax.set_ylabel('Ocupación')
plt.legend()
plt.savefig('Figures/ARIMA_pred.png')
plt.show()

real=y['2017':]
MAPE = (sum(abs((real['occ']-pred.predicted_mean))/
real['occ'])/len(real['occ']))*100
print(MAPE)

y2=y[2018:]
pred_uc = results.get_forecast(steps=30)
pred_ci = pred_uc.conf_int()
ax = y2.plot(label='observed', figsize=(14, 7))
pred_uc.predicted_mean.plot(ax=ax, label='Forecast')
ax.set_xlabel('Date')
ax.set_ylabel('occ')
plt.legend()
plt.savefig('Figures/ArimaPred.png',dpi=100)
plt.show()

arima_pred = pred_uc.predicted_mean
arima_pred.to_csv('data/arima_pred.csv', sep=',')








\end{verbatim}

	
	\chapter{Implementación de modelo de optimización de ingresos}
\label{ch:anexod}

\begin{verbatim}
import pandas as pd
import numpy as np
import scipy
import math as mt
import pickle as pkl
import os
import psycopg2
import matplotlib
import matplotlib.pyplot as plt
from numpy.random import normal
import calendar
from scipy.optimize import curve_fit
plt.rcParams['figure.figsize'] = (16,8)
import warnings
warnings.filterwarnings('ignore')
import plotly.plotly as py
import plotly.graph_objs as go
import plotly.figure_factory as ff
from plotly.offline import download_plotlyjs, 
init_notebook_mode, plot, iplot
init_notebook_mode(connected=True)
from datetime import datetime
from datetime import timedelta

# Cargamos los resultados de la regresion poisson para pronosticar ocupacion
data = pd.read_csv("data/forecasted_demand.csv")
date_start = min(data.dia.values)
date_end = max(data.dia.values)
date_end=datetime.strptime(date_end, '%Y-%m-%d')
date_end= date_end +  timedelta(days=1)
nb_days = (pd.to_datetime(date_end) - pd.to_datetime(date_start)).days
forecasted_demand = data.cn_predic.values
forecasted_demand=pd.DataFrame(forecasted_demand,
                               index=pd.date_range(start=date_start,
                                                   end=date_end,closed='left'),
                               columns=['occupancy'])
forecasted_demand.occupancy=data.cn_predic.values

def demand_price_elasticity(price, nominal_demand, elasticity=-2.0, nominal_price=120.0):      return nominal_demand * ( price / nominal_price ) ** (elasticity)

import scipy.optimize as optimize

def objective(p_t, nominal_demand=np.array([50,40,30,20]),
              elasticity=-2.0, nominal_price=1200.0):
    return (-1.0 * np.sum( p_t * demand_price_elasticity(p_t, 
            nominal_demand=nominal_demand,
            elasticity=elasticity,
            nominal_price=nominal_price) )) / 100

def constraint_1(p_t):
    return p_t


def constraint_2(p_t, capacity=20, forecasted_demand=35.0,
                 elasticity=-2.0, nominal_price=1200.0):
    return capacity - demand_price_elasticity(p_t, 
                            nominal_demand=forecasted_demand,
                            elasticity=elasticity,
                            nominal_price=nominal_price)

capacities = [20.0, 40.0, 60.0, 80.0]

optimization_results = {}
for capacity in capacities:
    nominal_price = 1200.0
    nominal_demand = forecasted_demand['occupancy'].values
    elasticity = -2.0

    p_start = 1250.0 * np.ones(len(nominal_demand))

    bounds = tuple((100.0, 2000.0) for p in p_start)

    constraints = ({'type': 'ineq', 'fun':  lambda x:  constraint_1(x)},
               {'type': 'ineq', 'fun':  lambda x, capacity=capacity,
                                           forecasted_demand=nominal_demand,
                                           elasticity=elasticity,
                                           nominal_price=
                                           nominal_price: constraint_2(x,
                                           capacity=capacity,
                                           forecasted_demand=nominal_demand,                                                           elasticity=elasticity,
                                           nominal_price=nominal_price)})

    opt_results = optimize.minimize(objective, p_start,
                                    args=(nominal_demand,
                                    elasticity,
                                    nominal_price),
                                    method='SLSQP', bounds=bounds,
                                    constraints=constraints)

    optimization_results[capacity] = opt_results

time_array = np.linspace(1,len(nominal_demand),len(nominal_demand))
rate_df = pd.DataFrame(index=time_array)

for capacity in optimization_results.keys():
    rate_df = pd.concat([rate_df,
                         pd.DataFrame(optimization_results[capacity]['x'],
                                      columns=['{}'.format(capacity)],
                                      index=time_array)],axis=1)

rate_df.index.name = 'Day'
datelist = pd.date_range(start=date_start, end=date_end, closed='left').tolist()
rate_df.index = [ x.date() for x in datelist]

rate_df.to_csv('data/rates.csv')


\end{verbatim}



	%%%%%%%%%%%%%%%%%%%%%%%%%%%%%%%%%%%%%%%%%%%%%%
	% BIBLIOGRAPHY
	%%%%%%%%%%%%%%%%%%%%%%%%%%%%%%%%%%%%%%%%%%%%%%
	\clearpage
	\addcontentsline{toc}{chapter}{References} %Añadimos la bibliografia a la lista de contenidos.
	
	%%%%%%%%% Referencias usando el sistema embedido %%%%%%%%%%%
	% e.g. (Ejemplo tomado de https://en.wikibooks.org/wiki/LaTeX/Bibliography_Management)
	%
	% \begin{thebibliography}{9}
	%
	%	\bibitem{lamport94}
    %			Leslie Lamport,
    %			\emph{LaTeX: a document preparation system},
    %			Addison Wesley, Massachusetts,
  	%			2nd edition,
    % 			1994.
    %
	% \end{thebibliography}

	%%%%%%%%% Referencias usando bibtex %%%%%%%%%%%
	\bibliographystyle{plain}
	\bibliography{references} 

\end{document}